
\documentclass{icmart}
%\usepackage{latexsym,color,amsmath,amsthm,amssymb,amscd,amsfonts,mathrsfs,array,xfrac}
\usepackage{cite}
\usepackage{breqn}
\usepackage{tikz}
\usetikzlibrary{arrows,calc}
\tikzset{
%Define standard arrow tip
>=stealth',
%Define style for different line styles
help lines/.style={dashed, thick}, axis/.style={<->}, important
line/.style={thick}, connection/.style={thick, dotted}, }



\contact[ostrover@post.tau.ac.il]{School of Mathematical Sciences, 
Tel Aviv University, Ramat Aviv 69978 Israel}


\newtheorem{theorem}{Theorem}[section]
\newtheorem{corollary}[theorem]{Corollary}
\newtheorem{lemma}[theorem]{Lemma}
\newtheorem{proposition}[theorem]{Proposition}
\newtheorem{conjecture}[theorem]{Conjecture}
\newtheorem{question}[theorem]{Question}

\newtheorem*{coro}{Corollary} 


\theoremstyle{definition}
\newtheorem{definition}[theorem]{Definition}
\newtheorem{remark}[theorem]{Remark}


\newcommand{\Hom}{\operatorname{Hom}}
\newcommand{\Ker}{\operatorname{Ker}}


%\title[Ceci n'est pas un titre]{Ceci n'est pas un titre}
\title[When Symplectic Topology Meets  Banach Space Geometry]{When  Symplectic Topology Meets  Banach Space Geometry}


\author[Yaron Ostrover]
{Yaron Ostrover
%\thanks{Authors are grateful to some institution
%for its hospitality during the writing of this paper.}
}

\begin{document}

\begin{abstract}
%In this survey paper we present some recent works that take first steps toward establishing novel interrelations between symplectic geometry and 
%other fields of mathematics, namely: asymptotic geometric analysis, convex geometry, functional analysis, and the theory of normed function-spaces. 
In this paper we survey some recent 
works that take the first steps toward establishing  
bilateral connections between symplectic geometry and 
several other fields, namely,  asymptotic geometric analysis, classical convex geometry, and the theory of normed spaces. 

%These are 
% , and the theory of normed function-spaces. 
\end{abstract}

\begin{classification} 53D35,  52A23, 52A40, 37D50, 57S05.
\end{classification}

\begin{keywords}
Symplectic capacities, Viterbo's volume-capacity conjecture, %Symplectic  isoperimetry, 
Mahler's conjecture, Hamiltonian diffeomorphisms, Hofer's metric. 
\end{keywords}

\maketitle

\section{Introduction}


 In the last three decades, 
symplectic topology has had an astonishing amount of  fruitful interactions with other fields of mathematics, including complex and algebraic geometry,  dynamical systems, 
Hamiltonian PDEs, transformation groups, 
and low-dimensional topology; 
as well as with physics, where, for example, symplectic topology plays a key role in the  rigorous formulation of mirror symmetry. 

\smallskip

In this survey paper, we present some recent works that take first steps toward establishing novel interrelations between symplectic geometry and 
several fields of mathematics, % which hardly had any previous connection to it, 
namely, asymptotic geometric analysis, classical convex geometry,  and the theory of normed spaces. 
%expanding on both sides the methodological palette and the range of research questions that can be attacked
%\smallskip
In the first part of this paper (Sections~\ref{SEC:ISOPERIMTERIC} and~\ref{SEC:MAHLER}) we concentrate on the theory of symplectic measurements, 
which arose from the foundational work of Gromov~\cite{Gr} on pseudoholomorphic curves;  followed by the seminal works of Ekeland and Hofer~\cite{EH} and Hofer and Zehnder~\cite{HZ1} on variational theory 
in Hamiltonian systems, and Viterbo on generating  functions~\cite{V1}. This theory -- also known as the theory of ``symplectic capacities" -- lies nowadays at the core of symplectic geometry and topology.

\smallskip

In Section~\ref{SEC:ISOPERIMTERIC}, we focus on an open symplectic isoperimetric-type conjecture proposed  by Viterbo in~\cite{V}. It states that among all convex domains 
 with a given volume in the classical phase space ${\mathbb R}^{2n}$, the Euclidean ball has the maximal ``symplectic size" (see Section~\ref{SEC:ISOPERIMTERIC} below for the precise statement). 
In a collaboration with S. Artstein-Avidan and V. D.  Milman~\cite{AMO},   
we were able to prove an asymptotic version of Viterbo's conjecture, that is, we proved the conjecture up to a universal (dimension-independent) constant. This has been achieved by  adapting %and implementing 
techniques from asymptotic geometric analysis and adjusting them
to a symplectic context, while working exclusively in the linear symplectic category.

\smallskip

The fact that one can get within a constant factor to the full conjecture using only linear embeddings is somewhat surprising from the symplectic-geometric point of view, as in symplectic geometry one typically 
needs highly nonlinear tools to estimate capacities.
However, this  fits perfectly into the philosophy of asymptotic geometric analysis.
Finding dimension independent estimates is a
frequent goal in this field, 
where surprising phenomena such
as concentration of measure (see e.g.~\cite{MilSch}) imply the existence of order and structures
in high dimensions, despite the huge complexity it involves. 
It would be interesting to explore whether similar phenomena also exist in the framework of symplectic geometry.
A natural important source for the study of the asymptotic  behavior (in the dimension) of symplectic invariants is the field of statistical mechanics,  
where one considers systems with a large number of particles, and the dimension of the phase space is twice the number of degrees of freedom. 
%To the best of the author's knowledge, 
It seems that symplectic measurements were overlooked in this context so far. 


In Section~\ref{SEC:MAHLER} we go in the opposite direction: we show how symplectic geometry could potentially be used to tackle  a 70-years-old fascinating open question
in convex geometry, %which is a 70-years-old open problem 
known as the Mahler conjecture.
Roughly speaking, Mahler's conjecture states that the minimum of the product of the volume of a centrally symmetric convex body and the volume of its polar body is attained (not uniquely) for the hypercube. 
%This is not just a purely theoretical question; a positive answer would have %significant  consequences in convex geometry and asymptotic geometric analysis, as well as  applications in various fields such as geometry of numbers and theoretical computer science.
In a collaboration with S. Artstein--Avidan and R. Karasev~\cite{AKO}, we combined tools from symplectic geometry, classical convex analysis, and the theory of mathematical billiards,
%to connect 
and established a close relation between Mahler's conjecture and the above mentioned symplectic isoperimetric conjecture by Viterbo. More preciesly, we showed that Mahler's conjecture is equivalent to a special case of Viterbo's conjecture (see Section~\ref{SEC:MAHLER} for details). 
%This result opens a promising new direction toward solving Mahler's long-standing conjecture using symplectic methods.



In the second part of the paper (Section~\ref{SEC:HOFER}),  we explain how methods from functional analysis can be used to address  questions regarding the 
% we consider  the 
 geometry of the group ${\rm Ham}(M, \omega)$ of Hamiltonian
diffeomorphisms   associated with a symplectic manifold $(M,\omega)$. 
One of the most striking facts regarding
this group, discovered by Hofer in~\cite{H}, is that it carries an intrinsic geometry given by a Finsler bi-invariant metric, nowadays 
known as Hofer's metric. This metric measures the time-averaged minimal oscillation of a
Hamiltonian function that is needed to generate a Hamiltonian diffeomorphism starting from the identity. %physical motion. 
Hofer's metric has been
intensively studied in the past twenty years, leading to many discoveries covering a wide
range of subjects from Hamiltonian dynamics to symplectic topology (see e.g.,~\cite{HZ, Mcd1,P1} and the references therein).
A long-standing question raised by Eliashberg and Polterovich in~\cite{EliP} is whether Hofer's metric is the only bi-invariant Finsler metric on the group ${\rm Ham}(M, \omega)$. %of Hamiltonian diffeomorphisms.
Together with L. Buhovsky~\cite{BO}, and based on previous results by Ostrover and Wagner~\cite{OW}, we used methods from functional analysis and the theory of normed function spaces to  affirmatively answer this question. % give an affirmative answer to this question. 
We proved that any non-degenerate bi-invariant Finsler metric on ${\rm Ham}(M, \omega)$, which is
generated by a norm that is continuous in the $C^{\infty}$-topology, gives rise to the same
topology on ${\rm Ham}(M, \omega)$ as the one induced by Hofer's metric.
%We believe %it is likely 
%that the methods developed in~\cite{OW} and~\cite{BO} could be implemented in various other setups, and used to tackle
%different questions regarding geometric properties of Hofer's metric.

%the geometry of ${\rm Ham}(M, \omega)$.

\smallskip
As mentioned before, the outlined interdisciplinary connections described above  are just the first few steps in what seems to be a promising new direction. We hope that further exploration of these connections will strengthen the dialogue between these fields and symplectic geometry, and expand the range of methodologies alongside research questions that can be tackled through these means.
%We hope that further exploration of the above mentioned interdisciplinary connections  will strengthen the dialogue between the fields and add to the methodological pallet on both sides and expand the range of research questions that can be tackled through these means.
%We hope that....further exploration .... 
%
%
%expanding on both sides the methodological palette and the range of research questions that can be attacked.
%
%strengthen the dialogue between symplectic geometry and asymptotic
%geometric analysis, and very likely will lead to exciting new result in these two disciplines.

%regarding behavior of Hamiltonian diffeomorphisms
%(insert sentence about these methods' potential applications).
%\medskip
%
%In Section~\ref{SEC:HOFER} we consider a remarkable  bi-invariant metric, known as Hofer's metric, on the group of Hamiltonian diffeomorphism associated with a closed symplectic manifold $(M,\omega)$. 
%Intuitively, Hofer's metric measures the minimal time-averaged oscillation of a Hamiltonian function, that is needed to generate a
%symplectic map starting from the identity. Hofer's metric has been intensively studied in the past twenty years, leading to many discoveries that cover a wide range of subjects from
%Hamiltonian dynamics to symplectic topology
%
%\medskip
%Methods from Functional analysis, and from the theory of normed function-spaces to show the uniquness of Hofer's metric. 


\smallskip

We end this paper with several open questions and speculations regarding some of the mentioned topics (see Section~\ref{SEC:OQ}).
%Finally, in Section~\ref{SEC:OQ} open questions and speculations

\section{A Symplectic Isoperimetric Inequality} \label{SEC:ISOPERIMTERIC}



A classical result in symplectic geometry (Darboux's theorem) states that symplectic manifolds - in a sharp contrast
to Riemannian manifolds - have no local invariants (except, of course, the dimension). The first examples
of global symplectic invariants were introduced by Gromov in his seminal paper~\cite{Gr}, where
he developed and used pseudoholomorphic curve techniques to prove a striking symplectic rigidity result.
Nowadays known as Gromov's ``non-squeezing theorem", this result states that one cannot map a ball inside a thinner cylinder
by a symplectic embedding. %This theorem shows that symplectomorphisms are not nearly as malleable as  volume preserving transformations. 
This theorem paved the way to the introduction
of global symplectic invariants, called symplectic capacities which, roughly speaking, measure the
symplectic size of a set.


\smallskip


We will focus here on the case of the classical phase space ${\mathbb R}^{2n} \simeq {\mathbb C}^n$  equipped with the standard symplectic structure $\omega=dq \wedge dp$. 
We denote by $B^{2n}(r)$ the Euclidean ball of radius $r$, and by $Z^{2n}(r)$ the cylinder $B^{2}(r) \times {\mathbb C}^{n-1}$. 
Gromov's non-squeezing theorem asserts that if $r < 1$ there is no symplectomorphism $\psi$ of ${\mathbb R}^{2n}$ such that $\psi(B^{2n}(1)) \subset Z^{2n}(r)$.
The following definition, which crystallizes the notion of  ``symplectic size", was given by Ekeland and Hofer in their influential paper~\cite{EH}.



%\medskip
%
%More precisely, equip ${\mathbb R}^{2n} \simeq {\mathbb C}^n$ with the standard symplectic structure $\omega=dx \wedge dy$, and denote by $B^{2n}(r)$ the Euclidean ball of radius $r$, and by $Z^{2n}(r)$ the cylinder $B^{2}(r) \times {\mathbb C}^{n-1}$. 
%%\begin{definition} 

\smallskip


\noindent {\bf Definition:} 
A symplectic capacity on $({\mathbb R}^{2n},\omega)$ associates
to each  subset $U \subset {\mathbb R}^{2n}$ a number $c(U) \in
[0,\infty]$ such that the following three properties hold:

\smallskip


\noindent (P1) $c(U) \leq c(V)$ for $U \subseteq V$ (monotonicity);

\smallskip


\noindent (P2) $c \big (\psi(U) \big )= |\alpha| \, c(U)$ for  $\psi
\in {\rm Diff} ( {\mathbb R}^{2n} )$ such that $\psi^*\omega=
\alpha  \omega$ (conformality);

\smallskip


\noindent (P3) $c \big (B^{2n}(r) \big ) = c \big (Z^{2n}(r) 
 \big ) = \pi r^2$ (nontriviality and
normalization).
%where $B^{2k}(r)$ is the open $2k$-dimensional ball of radius $r$.
%\end{definition}

\smallskip

Note that (P3)  disqualifies any volume-related
invariant, while (P1) and (P2) imply that for $U, V \subset {\mathbb
R}^{2n}$, a necessary condition for the existence of a
symplectomorphism $\psi $ %  \in {\rm Symp}({\mathbb R}^{2n})$
 with $\psi(U) = V$, is $c(U) =c(V)$ for any symplectic capacity $c$.

%
%Note that the third property disqualifies any volume-related
%invariant, while the first two imply that for $U, V \subset {\mathbb
%R}^{2n}$, a necessary condition for the existence of a
%symplectomorphism $\psi $ %  \in {\rm Symp}({\mathbb R}^{2n})$
% with $\psi(U) = V$, is $c(U) =c(V)$ for any symplectic capacity $c$.

\smallskip

It is a priori unclear that symplectic capacities exist.
The above mentioned non-squeezing result  naturally leads to the definition of two symplectic capacities: 
the Gromov radius, defined by $\underline c(U)=\sup\{\pi r^2 \, | \, B^{2n}(r) \stackrel{\rm s} \hookrightarrow U \} $; and the 
cylindrical capacity, defined by $\overline c(U) = \inf\{\pi r^2  \, | \, U \stackrel{\rm s} \hookrightarrow Z^{2n}(r) \} $, where $\stackrel{\rm s} \hookrightarrow$ stands for symplectic embedding. 
It is easy to verify that these two capacities are the smallest and largest  possible symplectic capacities, respectively. 
Moreover, it is also known that the existence of a single capacity readily implies Gromov's non-squeezing theorem, as well as 
the Eliashberg-Gromov $C^0$-rigidity theorem, which states that for any closed symplectic manifold $(M,\omega)$,  the  symplectomorphism group ${\rm Symp}(M,\omega)$  is $C^0$-closed in the group  of all  diffeomorphisms of $M$  (see e.g.,  Chapter 2 of~\cite{HZ}). 

\smallskip



Shortly after Gromov's work, other symplectic capacities were constructed,
such as the Hofer-Zehnder~\cite{HZ} and the Ekeland-Hofer~\cite{EH} capacities, the displacement energy~\cite{H},  the Floer-Hofer capacity~\cite{FH,FHW},
spectral capacities~\cite{FGS,Oh,V1}, and, more recently, Hutchings's embedded contact homology capacities~\cite{Hu1}.
Nowadays, symplectic capacities are among the most fundamental objects in symplectic geometry,
and are the subject of intensive research efforts (see e.g.,~\cite{Hu2,IrirKei1, L, LMT, LMS, Lu, Mcd2,McSch,Schle}, and~\cite{CHLS} for a recent detailed survey and more references).
% However, i
 However, in spite of the rapidly accumulating
knowledge regarding symplectic capacities, they are %still 
notoriously difficult
 to compute, and
there are no general methods even to effectively estimate them.
% 
% In spite of their significance and vast
%array of applications to symplectic geometry and topology, symplectic capacities are notoriously difficult
%to compute, and there are no general methods to effectively estimate them. 
%% Consequently, many
%%fundamental questions regarding these capacities still remain unanswered.

\smallskip


%Throughout, by a convex body we shall mean a convex bounded set in R2n with non-empty interior.





In~\cite{V}, Viterbo investigated the relation between the
symplectic way of measuring the size of sets using symplectic
capacities, and the classical  approach using  volume. Among many other inspiring results, in that work he
conjectured that in the class of convex bodies in ${\mathbb R}^{2n}$
with fixed volume, the Euclidean ball $B^{2n}$ maximizes any
given symplectic capacity. More precisely,


\begin{conjecture}[Viterbo's volume-capacity inequality conjecture] \label{iso-per-conj} For
any convex body $K $ in ${\mathbb R}^{2n}$ and any symplectic
capacity $c$,
\begin{equation*} \label{AAMO-result} {\frac {c(K)} {c(B)}} \leq  \left  (   {\frac {{\rm Vol}(K)}
{{\rm Vol}(B)}} \right )^{1/n}, \ \ {\rm where} \ B = B^{2n}(1).
\end{equation*}
\end{conjecture}

Here and henceforth a convex body of ${\mathbb R}^{2n}$ is a compact convex set with non-empty interior. 
The isoperimetric inequality above was proved in~\cite{V} up to a
constant that depends linearly on the dimension using the classical John ellipsoid theorem. In a joint work
with S. Artstein-Avidan and V. D. Milman (see~\cite{AMO}), we made further progress towards the proof
of the conjecture. By customizing %powerful 
methods and
techniques from asymptotic geometric analysis and adjusting them to
the symplectic context, we were able to prove Viterbo's conjecture up
to a universal  (i.e., dimension-independent) constant. More
precisely, we proved that %in~\cite{AMO} %that:
% \medskip A joint work of the author with S. Artstein-Avidan and V.
% Milman (see~\cite{AO} and~\cite{AMO}) brought together for the first
% time tools and ideology from two different fields: Symplectic
% Geometry on the one hand, and Asymptotic Geometric Analysis on the
% other, to tackle the above conjecture.
%  %to arrive at some new results. In particular, substantial
% %progress has been made toward the proof of the above conjecture.
% More precisely, the following theorem, which signifucaly improves a
% previous result by Viterbo~\cite{V1}, gives a dimension-independent
% bound for the symplectic capacity of a convex
%  domain by its volume radius,
% %, was proven in~\cite{AMO}.
\begin{theorem} \label{up-to-uni-constnat} There is a universal constant $A$ such that for
any convex domain $K$ in ${\mathbb R}^{2n}$, and any symplectic
capacity $c$, one has
$$ {\frac {c(K)} {c(B)}} \leq A \, \left (   {\frac {{\rm Vol}(K)} {{\rm Vol}(B)}} \right )^{1/n}, \ \ {\rm where} \ B = B^{2n}(1).$$
\end{theorem}

We emphasize that %the method of proof 
in the proof of Theorem~\ref{up-to-uni-constnat}  we work exclusively in the category of linear symplectic geometry.
It turns out that even in this limited category of linear symplectic transformations, there are tools which are powerful enough to obtain a dimension-independent estimate as above.
While this fits with the philosophy of asymptotic geometric analysis, it is less expected from a
symplectic geometry point of view, where  one expects that %to require
highly nonlinear methods, such as folding and wrapping techniques (see e.g., the book~\cite{Schle}), would be required to effectively estimate symplectic capacities.

\smallskip

The proof of Theorem~\ref{up-to-uni-constnat} above is based on two ingredients. The first is the following simple geometric observation (see Lemma 3.3 in~\cite{AMO}, cf.~\cite{APB}).
\begin{lemma} \label{lem-complex-symetric} 
If a convex body  $K \subset {\mathbb C}^{n}$  satisfies $K=iK$, then  $\overline c(K) \leq {\frac 4 {\pi}} \, \underline c(K)$.
\end{lemma}
\begin{proof}[Sketch of Proof]
Let $rB^{2n}$ be the  largest multiple  of the unit ball contained in $K$, and let $x  \in \partial K \cap rS^{2n-1}$ be a contact point between the boundary of $K$ and the boundary of  $rB^{2n}$.
It follows from the convexity assumption that the body $K$ lies between the hyperplanes $x + x^{\perp}$
and $-x + x^{\perp}$. Moreover,  since $K=iK$, it lies also between $-ix + ix^{\perp}$ and $ix +
ix^{\perp}$. Thus, the projection of $K$ onto the plane spanned
by $x$ and $ix$ is contained in a square of edge length $2r$. This square  can be turned into a disc with area $4r^2$, after applying a non-linear
symplectomorphism which is essentially two-dimensional.  Therefore, $K$
is contained in a symplectic image of the cylinder $Z^{2n}(\sqrt{4/{\pi}}\,r)$, and the lemma follows.
%$Z^{2n}(\sqrt{{\frac 4 {\pi}} }r)$, and the lemma follows.
%in turn is contained in a disc of radius $\sqrt{2}r$. Therefore $K$
%is contained in a cylinder of radius $\sqrt{2}r$ with base spanned
%by $x$ and $ix$. Since this cylinder is a unitary image of the
%standard symplectic cylinder $Z^{2n}(\sqrt{2}r)$, the lemma follows.
\end{proof}
Since by monotonicity, Conjecture~\ref{iso-per-conj} trivially holds for the Gromov radius $ \underline c$, it follows from Lemma~\ref{lem-complex-symetric}  that
%
%
%Combining Lemma~\ref{lem-complex-symetric} with the fact that %by the monotonicity property of  capacities, 
%Conjecture~\ref{iso-per-conj} holds for the Gromov radius $ \underline c$, we conclude that 
\begin{corollary} \label{COR:about-sym-bodies}
Theorem~\ref{up-to-uni-constnat} holds for 
convex bodies $K \subset {\mathbb C}^{n}$ such that $K=iK$.  

\end{corollary}
\smallskip


The second ingredient in the proof is  a profound result in asymptotic geometric analysis  discovered by V.D. Milman in the mid 1980's called the ``reverse Brunn-Minkowski inequality" (see~\cite{Mil,Milm1317}).
Recall that the classical Brunn-Minkowski
inequality states that if $A$ and $B$ are non-empty Borel
subsets of ${\mathbb R}^n$, then
$$ {\rm Vol} (A+B)^{1/n} \geq {\rm Vol}(A)^{1/n} + {\rm
Vol}(B)^{1/n},$$ 
where $A+B = \{x+y \, | \, x \in A, \, y \in B \}$ is the Minkowski sum. 
Although at  first glance it seems that one cannot
expect any inequality in the reverse direction (consider, e.g.,  two very long and thin ellipsoids pointing in orthogonal
directions in ${\mathbb R}^2$), it turns out that for convex bodies, if one allows for an extra choice of
``position'', i.e., a volume-preserving linear image of the bodies, then one can reverse the Brunn-Minkowski inequality up to a universal constant factor. %More precisely, 

%a reverse inequality is possible. %More precisely, 

\begin{theorem}[Milman's reverse Brunn-Minkowski inequality] \label{RBMIneq}
For any two convex bodies $K_1,K_2$ in ${\mathbb R}^n$, there exist linear volume preserving
transformations $T_{K_i}$ $(i=1,2)$, such that  for $\widetilde K_i = T_{K_i}(K_i)$ one has
$$ {\rm Vol} (\widetilde K_1+ \widetilde K_2)^{1/n} \leq  C \left ( {\rm Vol}(\widetilde K_1)^{1/n} + {\rm Vol}(\widetilde K_2)^{1/n} \right),$$
for some absolute constant $C$.
\end{theorem}
We emphasize that the transformation $T_{K_i} (i=1,2)$ in Theorem~\ref{RBMIneq} depends solely on the body $K_i$, and not on the joint configuration of the bodies $K_1$ and $K_2$. For more details on the reverse Brunn-Minkowski inequality see~\cite{Milm1317,P1}.

\smallskip

We can now sketch the proof of Theorem~\ref{up-to-uni-constnat} (for  more details see~\cite{AMO}). Since every symplectic capacity is bounded above by the cylindrical capacity  $\overline c$, it is enough to prove the theorem for $\overline c$. For the sake of simplicity, we assume in what follows that $K$ is centrally symmetric, i.e., $K=-K$.  This assumption is not too restrictive, since by a classical result of Rogers and Shephard~\cite{RogShe} one has that ${\rm Vol}(K+(-K)) \leq 4^n {\rm Vol}(K)$. 
After adjusting Theorem~\ref{RBMIneq} to the symplectic context, one has that for
any convex body $K \subset {\mathbb R}^{2n}$, there exists a linear symplectomorphism $S \in {\rm Sp}(2n)$ such that $SK$ and $iSK$ satisfy
the reverse Brunn-Minkowski inequality, that is, the volume  ${\rm Vol}(SK + iSK)$ is less than some constant times ${\rm Vol}(K)$. 
Combining this with the properties of symplectic capacities and Corollary~\ref{COR:about-sym-bodies}, 
%
%
 %the fact that from Lemma~\ref{lem-complex-symetric}, it follows that Theorem~\ref{up-to-uni-constnat} holds for 
%convex domains $K \subset {\mathbb R}^{2n}$ satisfying the symmetry assumption $K=iK$. 
%
we conclude that 
$$  {\frac {\overline c(K)} {\overline c(B)}}  \leq  {\frac {\overline c(SK+iSK)} {\overline c(B)}}   \leq A \, \left  (   {\frac {{\rm Vol}(SK+iSK)} {{\rm Vol}(B)}} \right )^{\frac 1 n}  \leq  A'  \, \left (   {\frac {{\rm Vol}(K)} {{\rm Vol}(B)}} \right )^{\frac 1 n} ,$$
for some universal constant $A'$, and thus Theorem~\ref{up-to-uni-constnat}  follows.   


%
%Moreover, from Lemma~\ref{lem-complex-symetric} it follows that 
%$ {\overline c(SK + iSK)} \leq  {\frac {\pi} 4 }{\underline c(SK + iSK)}$.
%
%$$  {\frac {\overline c(K)} {\overline c(B)}}     \leq  {\frac {\overline c(SK + iSK)} {\overline c(B)}}  \leq   {\frac {\pi} 4 } {\frac {\underline c(SK + iSK)} {\overline c(B)}}   \leq A \, \Bigl  (   {\frac {{\rm Vol}(K)} {{\rm Vol}(B)}} \Bigr )^{1/n} $$
%

%
%we show that every convex body
%K has a linear symplectic image K0 = SK such that the couple K0 and iK0
%satisfy
%the inverse Brunn-Minkowski inequality.
\smallskip

%
%It was discovered in \cite{Mil} that one can reverse the
%Brunn-Minkowski inequality, up to a universal constant factor, as
%follows: for every convex body $K$ there exists a linear
%transformation $T_K$, which is volume preserving, such that for any
%two bodies $K_1$ and $K_2$, the bodies $T_{K_1}K_1$ and $T_{K_2}K_2$
%satisfy an inverse Brunn-Minkowski inequality up to some universal
%constant. 
%given any symmetric convex set $K$, there exists a linear symplectomorphism $T$ such that the volume of  $T(C) + i T(C)$ is less than constant ? vol(C),
%
%\smallskip
%

%Is there a quantitative refinement of the Weinstein conjecture?
%
%\smallskip
%
%
%Finding dimension independent estimates is a
%frequent goal in asymptotic geometric analysis, where surprising phenomena such
%as concentration of measure (see e.g.~\cite{MilSch}) imply the existence of order and structures
%in high dimension, despite the huge complexity it involves. It is encouraging to see
%that such phenomena also exist in Symplectic Geometry, and although this is just a first example, we hope more will follow.
%
%
%STATISTICAL MECHANICS
%\smallskip



In the next section we will show a surprising connection between Viterbo's volume-capacity conjecture and a seemingly remote open conjecture from the field of convex geometric analysis: 
the Mahler conjecture on the volume product of centrally symmetric convex bodies.

%Finally we remark that the question whether the constant $A$ in Theorem~\ref{up-to-uni-constnat} equals one remains open, and
%is a work in progress.

%\section{From Viterbo's Conjecture to Mahler's Conjecture via Billiard Dynamics }
%\section{ Mahler's Conjecture, and Billiard Dynamics }
\section{ A Symplectic View on Mahler's Conjecture} \label{SEC:MAHLER}



Let $(X,\| \cdot \|)$ be an $n$-dimensional normed space and let
$(X^*,\| \cdot \|^*)$  be its dual space. Note that the product space $X
\times X^*$ carries a canonical symplectic structure, given by the
skew-symmetric bilinear form $\omega \bigl ( (x,\xi),(x',\xi') \bigr
) = \xi(x')-\xi'(x)$, and a canonical volume form, the {\it
Liouville} volume, given by $ \omega^n/n!$. A fundamental question
in the field of convex geometry, raised by Mahler in~\cite{Ma}, is to find  upper and lower bounds for the
Liouville volume of $B \times B^{\circ} \subset X \times X^*$, where
$B$ and $B^{\circ}$ are the unit balls of $X$ and $X^*$,
respectively. In what follows we shall denote this volume by
$\nu(X)$. 
The quantity $\nu(X)$ is an affine invariant of $X$, i.e. it
is invariant under invertible linear transformations. We remark that
in the context of convex geometry $\nu(X)$ is also known as the
{\it Mahler volume} or the {\it volume product} of $X$.

\smallskip




The Blaschke-Santal\'o inequality asserts that the maximum of
$\nu(X)$ is attained if and only if $X$ is a Euclidean space. This
was proved by Blaschke~\cite{Bl} for dimensions two and three, and
generalized by Santal\'o~\cite{Sa} to higher dimensions. 
%The case of equality was later
%characterized by  Saint-Raymond~\cite{SR} (cf.~\cite{MP}).
The following sharp lower bound for $\nu(X)$ was
conjectured by Mahler~\cite{Ma} in 1939:


%\medskip

\begin{conjecture}[Mahler's volume product conjecture]\label{Mahler-conj}
For any
$n$-dimensional normed space $X$ one has $\nu(X) \geq 4^n/n!.$ 
\end{conjecture}
%
%\medskip
%
%\noindent {\bf Mahler Conjecture:} \label{Mahler-Conj} For an
%$n$-dimensional normed space $X$ one has $\nu(X) \geq 4^n/n!$ 



The conjecture has been verified by Mahler~\cite{Ma} in the
two-dimensional case. In higher dimensions it is proved only in
a few special cases (see e.g.,~\cite{GMR,Kim,ME1,NPRZ, R1,R2, RSW,SR, St}). 
%
%
%(c.f.~\cite{GMR}), when $X$ has a 1-unconditional
%basis~\cite{ME1,SR,R2}, and when the unit ball of
%$X$ is sufficiently close to the unit cube in the Banach--Mazur
%distance~\cite{NPRZ}. % add reference to Weberndorfer; see artem and his talksin banff
A major breakthrough towards answering Mahler's conjecture is a result 
due to Bourgain and Milman~\cite{BM}, who used sophisticated tools from functional analysis to
show that the conjecture holds asymptotically, i.e., up to a factor
$\gamma^n$, where $\gamma$ is a universal constant. 
This result has been re-proved later on, with entirely different methods, by Kuperberg~\cite{Ku}, using differential geometry, 
and independently by Nazarov~\cite{Naz},  using the theory of functions of several complex variables. A new proof using simpler asymptotic geometric analysis tools has been recently discovered by Giannopoulos, Paouris, and Vritsiou~\cite{GPV}. 
The best known constant today, $\gamma = \pi/4$, is due to Kuperberg~\cite{Ku}.

%The conjecture has been verified by Mahler~\cite{Ma} in the
%two-dimensional case. In higher dimensions it is proved only in
%some very special cases, namely, when the unit ball of $X$ is a
%zonoid~\cite{R1} (c.f.~\cite{GMR}), when $X$ has a 1-unconditional
%basis~\cite{ME1,SR,R2}, and when the unit ball of
%$X$ is sufficiently close to the unit cube in the Banach--Mazur
%distance~\cite{NPRZ}. % add reference to Weberndorfer; see artem and his talksin banff
%The first major breakthrough towards answering Mahler's conjecture was a result 
%due to Bourgain and Milman~\cite{BM}, who  used sophisticated tools from functional analysis to
%show that the conjecture holds asymptotically, i.e., up to a factor
%$\gamma^n$, where $\gamma$ is a universal constant. 
%This result has been re-proved later on, by entirely different methods, by Kuperberg~\cite{Ku}, using differential geometry, 
%and independently by Nazarov~\cite{Naz},  using the theory of functions of several complex variables. A new proof using simpler asymptotic geometric analysis tools has been recently discovered by Giannopoulos, Paouris, and Vritsiou~\cite{GPV}. 
%The best known constant nowadays, $\gamma = \pi/4$, is due to Kuperberg~\cite{Ku}.
%%The best known
%%constant nowadays $\gamma = \pi/4$ is due to Kuperberg~\cite{Ku}
%%(cf. Nazarov~\cite{Naz} for a different approach). 



\smallskip

Despite great efforts to deal with the general case, a proof of Mahler's conjecture has been insistently elusive so far, and is currently the subject of intensive research.
%We remark that in
A possible reason for this, as pointed out for example by Tao in~\cite{Tao}, is that, in contrast with the above mentioned Blaschke-Santal\'o inequality, the equality case
in Mahler's conjecture, which is obtained for example for the space
$l^n_{\infty}$ of bounded sequences with the standard maximum norm,
is not unique, and there are in fact many distinct extremizers for the (conjecturally) lower bound of $\nu(X)$ (see, e.g., the discussion in~\cite{Tao}).
This practically renders impossible any proof based on currently known optimisation techniques, and a radically different approach seems to be needed. 
%
%
%``In my opinion, the main reason why this    conjecture is so difficult is that unlike the upper bound, 
% there are many distinct extremisers for the lower bound (Hanner polytopes).  
% It is really difficult to conceive of any sort of flow or optimisation   
%  procedure which would converge to exactly these bodies and  no others;   a radically different type of argument might be needed. "

%\smallskip
%It is interesting to note that from a symplectic point of view, the equality  minimum of the volume product of a centrally symmetric convex body and the volume of its polar is attained for the hypercube

\smallskip

We refer the reader to  Section~\ref{SEC:OQ} below for further discussion on the characterization of the equality case of Mahler's conjecture,  and its possible connection with symplectic geometry.

\smallskip

In a recent work with S. Artstein-Avidan and R. Karasev~\cite{AKO}, we combined  tools from symplectic geometry, convex analysis, and the theory of mathematical billiards, and  established a close relationship between 
Mahler's conjecture and Viterbo's volume-capacity conjecture. More precisely, we proved in~\cite{AKO} that
%\begin{theorem}[\cite{AKO}] Mahler's conjecture is equivalent to Viterbo's volume-capacity conjectured,
%where the latter is restricted to domains of the form $\Sigma \times \Sigma^{\circ} \subset {\mathbb R}^{2n}$, where $\Sigma \subset {\mathbb R}^n_q$ is a centrally symmetric convex body.  %in the classical phase space.
%\end{theorem}
\begin{theorem} \label{THM-V-M} Viterbo's volume-capacity conjecture implies Mahler's conjecture.
\end{theorem}
In fact, it follows from our proof that Mahler's conjecture is  equivalent to a special case of Viterbo's  conjecture, where the latter is restricted to the Ekeland-Hofer-Zehnder symplectic capacity, and to domains in the classical phase space of the form $\Sigma \times \Sigma^{\circ} \subset {\mathbb R}^{2n} = {\mathbb R}^n_q \times {\mathbb R}^n_p$   (for more details see~\cite{AKO}, and in particular Remark 1.9 ibid.). Here, $\Sigma \subset {\mathbb R}^n_q$ is a centrally symmetric convex body, the space ${\mathbb R}^n_p$ is identified with the dual space $({\mathbb R}^n_q)^*$, and $$\Sigma^{\circ} = \{ p \in  {\mathbb R}^n_p \, | \, p(q)\leq 1 \ {\rm for \ every } \ q \in \Sigma 
 \}$$  
 
 





Theorem~\ref{THM-V-M} is a direct consequence of % follows immediately from 
the following result proven in~\cite{AKO}.

\begin{theorem} \label{THM:CAP=4} There exists a symplectic capacity $c$  such that $ c(\Sigma \times \Sigma^{\circ})=4$ for every centrally symmetric convex body $\Sigma \subset {\mathbb R}^n_q$.
\end{theorem}

With Theorem~\ref{THM:CAP=4} at our disposal, it is not difficult to derive Theorem~\ref{THM-V-M}.
\begin{proof}[{\bf Proof of Theorem~\ref{THM-V-M}}]
Assume that Viterbo's volume-capacity conjecture holds. From Theorem~\ref{THM:CAP=4} it follows that there exists a symplectic capacity $c$ such that for every centrally symmetric convex body $\Sigma \subset {\mathbb R}^n_q$ one has
$$ {\frac {4^n} {\pi^n}} = {\frac { c^n(\Sigma \times \Sigma^{\circ})} {\pi^n}}
 \leq   {\frac {{\rm Vol}(\Sigma \times \Sigma^{\circ}) } {{\rm Vol}(B^{2n})} }  =  {\frac {n! \, {\rm Vol}(\Sigma \times \Sigma^{\circ})} {{\pi^n}}
},
$$
which  is exactly the bound for  ${\rm Vol}(\Sigma \times
\Sigma^{\circ})$ required by Mahler's conjecture.
\end{proof}

%Indeed, if Viterbo's conjecture holds, then from Theorem~\ref{THM:CAP=4} it follows that $$4^n= \widetilde c^n( \Sigma \times \Sigma^{\circ}) \leq n! \, {\rm Vol}(\Sigma \times \Sigma^{\circ})$$ 
In the rest of this section we sketch the proof of Theorem~\ref{THM:CAP=4} (see~\cite{AKO} for a detailed exposition). %
% and for the completeness of the presentation, we will   which was given in~\cite{AKO}.
We remark that an alternative proof, based on an approach to billiard dynamics developed in~\cite{BB}, %of K. Bezdek and D. Bezdek~\cite{BB}, 
was recently given in~\cite{ABKS}. We start with recalling the definition of %introducing 
the Ekeland-Hofer-Zehnder capacity, which is the symplectic capacity that appears in Theorem~\ref{THM:CAP=4}.

 
%
%\subsection{The Hofer-Zehnder Capacity and Minkowski Billiards } \label{sec-HZ-Mink-billiards}
%
%In this section we describe the relation established in~\cite{AAO1} between the Hofer--Zehnder capacity~\cite{HZ}, restricted to the class of convex domains, and the minimal length of periodic 
%Minkowski billiard trajectories.  For the reader's convenience,  we recall first some of the relevant definitions and notations. For a detailed exposition and proofs, see~\cite{AAO1}.

\smallskip 

The restriction of the standard symplectic form
$\omega=dq \wedge dp$ to a smooth closed connected 
hypersurface ${\mathcal S}
\subset {\mathbb R}^{2n}$ defines a 1-dimensional subbundle
${\rm ker}(\omega | {\mathcal S})$, whose integral curves comprise the
characteristic foliation of ${\mathcal S}$. In other words, a {\it closed
characteristic}  of  ${\mathcal S}$ is an embedded circle
in  ${\mathcal S}$ tangent to the canonical %characteristic 
line bundle
\begin{equation*} {\mathfrak S}_{{\mathcal S}} = \{(x,\xi) \in T
{\mathcal S} \ | \ \omega(\xi,\eta) = 0 \ {\rm for \ all} \ \eta \in T_x
{\mathcal S} \}. \end{equation*}
%The classical geometric problem of finding a closed characteristic %on $\Sigma$
%has a well-known dynamical interpretation: if  the boundary
%$\partial \Sigma$ is represented as a regular energy surface $\{x
%\in {\mathbb  R}^{2n} \ | \ H(x) = {\rm const} \}$ of a smooth
%Hamiltonian function $H : {\mathbb  R}^{2n} \rightarrow {\mathbb
%R}$, then the restriction to $\partial \Sigma$ of the Hamiltonian
%vector field $X_H$, defined by $i_{X_H} \omega_{\rm st} = -dH$, is a section
%of ${\mathfrak S}_{\Sigma}$. Thus, the image of the periodic
%solutions of the classical Hamiltonian equation $\dot x= X_H(x) = J\nabla H(x)$ on
%$\partial \Sigma$ are precisely the closed characteristics of
%$\partial \Sigma$. 
Recall that the symplectic action %$A(\gamma)$ 
of a closed curve $\gamma$ % which is the enclosed symplectic area, 
is defined by $A(\gamma) = \int_{\gamma} \lambda$,
where  $\lambda =pdq$ is the Liouville 1-form. %whose differential  $d \lambda = \omega$. 
The action spectrum of ${\mathcal S}$ is %defined as
\begin{equation*}  {\cal L}({\mathcal S}) = \left \{ \, | \, {A}({\gamma}) \,  | \, ;
\, \gamma \ {\rm closed \ characteristic \ on} \  {\mathcal S}
\right \}.\end{equation*}


The following theorem, which is a combination of results from~\cite{EH} and~\cite{HZ}, states that on the class of convex domains in ${\mathbb R}^{2n}$,  the  Ekeland-Hofer capacity $c_{_{\rm EH}}$ and Hofer-Zehnder capacity $c_{_{\rm HZ}}$ coincide, 
and are given by the minimal action over all closed characteristics on the boundary of the corresponding convex body. 
%
%
%which serves here also as the definition 
%of the Ekeland--Hofer--Zehnder capacity for the class of smooth convex bodies, 
%can be found, e.g., in~\cite{HZ}.
%is a combination of results from~\cite{EH} and~\cite{HZ}.
\begin{theorem} \label{Cap_on_covex_sets} Let $ K \subseteq {\mathbb
R}^{2n}$ be a convex bounded domain with smooth boundary. % $\partial K$. 
Then there exists at least one closed characteristic $\widetilde \gamma
\subset \partial K$ satisfying
\begin{equation*} 
 c_{_{\rm EH}}(K) = c_{_{\rm HZ}}(K)= { A}(\widetilde \gamma) =  \min {\cal
L}( \partial K). \end{equation*}
\end{theorem}

We remark that although the above definition of closed characteristics, as well as Theorem~\ref{Cap_on_covex_sets}, 
were given only for the class of convex bodies with smooth boundary, they can naturally be generalized 
to the class of convex sets in ${\mathbb R}^{2n}$ with non-empty interior  (see~\cite{AAO1}). In what follows, we
refer to the coinciding Ekeland-Hofer and Hofer-Zehnder capacities on this class as the Ekeland-Hofer-Zehnder capacity. % and denote it by $c_{_{\rm EHZ}}$. 

\smallskip

We turn now to show that for every centrally symmetric convex body $\Sigma \subset {\mathbb R}^n_q$, the Ekeland--Hofer--Zehnder capacity satisfies $c_{_{\rm EHZ}} (\Sigma \times \Sigma^{\circ})=4$.
For this purpose, we now switch gears and turn to mathematical billiards in Minkowski geometry. 

\smallskip


It is  folklore to people in the field that billiard flow can be treated, roughly speaking, as the limiting case of geodesic flow on a boundaryless manifold. 
Indeed, 
%
%  are the geodesic 
%ows on Riemannian manifolds with boundaries. They can also
%be treated as a limit case of the geodesic 
%ows on boundaryless manifolds, at least heuristically.
let $\Omega$ be a smooth plane billiard table, and consider its ``thickening", i.e. an infinitely thin three
dimensional body whose boundary $\Gamma$ is obtained by pasting two copies of $\Omega$ along their boundaries
and smoothing the edge. Thus, a billiard trajectory in $\Omega$ can be viewed as a geodesic line on
the boundary of $\Gamma$, that goes from one copy of $\Omega$ to another each time the billiard ball bounces off the boundary. 
The main technical difficulties  with this strategy is the rigorous treatment of the limiting process, and the analysis involved with the dynamics near the  boundary. 
One approach to  billiard dynamics and the existence question of periodic trajectories is an  approximation scheme which uses a certain ``penalization method"  developed by Benci and  Giannoni in~\cite{BG} (cf.~\cite{AM,IrieKei2}).
In what follows we present an alternative approach, and use characteristic foliation on singular  convex hypersurfaces in ${\mathbb R}^{2n}$ (see e.g.,~\cite{Cl,Eke,Kun}) to describe Finsler type billiards for convex domains in the configuration space ${\mathbb R}^n_q$.
The main advantage of this approach is that it allows one to use the natural one-to-one correspondence between the geodesic flow on a manifold  and the characteristic foliation on its unit cotangent bundle, and thus provides a natural  ``symplectic setup"  in which one can use tools 
such as Theorem~\ref{Cap_on_covex_sets} above in the context of billiard dynamics.  In particular, we show that the Ekeland-Hofer-Zehnder capacity of certain Lagrangian product configurations ${\mathcal K} \times {\mathcal T}$ in the classical phase space ${\mathbb R}^{2n}$  is the length of the shortest periodic  {\it  ${\mathcal T}$-billiard trajectory in ${\mathcal K}$} (see e.g.,~\cite{AAO1,V}), which we turn now to describe.
%We will follow~\cite{Kun,Rock} characteristic foliation on singular hypersurfaces with a certain type of Finsler type  billiard.  



\smallskip

The general study of billiard dynamics in Finsler  and Minkowski
geometries was initiated by Gutkin and Tabachnikov in~\cite{GT}. From the point of view of geometric optics, Minkowski billiard
trajectories describe the propagation of light in a homogeneous
anisotropic medium that contains perfectly reflecting mirrors. % (see~\cite{GT}).
Below, we focus on the special case of Minkowski billiards in a smooth convex
body  ${\mathcal K} \subset {\mathbb R}^n_q$. %Roughly speaking, 
We equip ${\mathcal K}$
with a metric given by a certain norm $\| \cdot \|$, and
consider billiards in ${\mathcal K}$ with respect to the geometry induced by $\| \cdot \|$.
More precisely, let ${\mathcal K} \subset {\mathbb R}^n_q$, and ${\mathcal T}  \subset {\mathbb R}^n_p$ be two convex bodies with smooth boundary, and consider the 
unit cotangent bundle
\begin{equation*}
U_{\mathcal  T}^*{\mathcal K} := {\mathcal K}  \times {\mathcal T} = \{ (q,p) \, | \, q \in {\mathcal K}, \ {\rm and} \
g_{\mathcal T} (p)  \leq 1 \} \subset  T^* {\mathbb R}^n_q  = {\mathbb R}^{n}_q
\times  {\mathbb R}^{n}_p. \end{equation*} Here $g_{\mathcal T}$ is the gauge function % of ${\mathcal T}$ i.e., 
$g_{\mathcal T}(x) = \inf \{r  \, | \, x \in r {\mathcal T} \}$. %In particular, 
When ${\mathcal T}=-{\mathcal T}$ is centrally symmetric  %i.e., ${\mathcal T}=-{\mathcal T}$,
  one has $g_{\mathcal T}(x) = \|x\|_{\mathcal T}$.
For $p \in \partial {\mathcal T}$, the gradient vector $\nabla g_{\mathcal T}(p)$ is the outer normal to $\partial {\mathcal T}$ at the point $p$,  and is naturally considered to be in $\mathbb R^n_q = (\mathbb R^n_p)^* $.
% where we have used the standard identification between $T_p{\mathbb R}^n$ and $T^*_p{\mathbb R}^n$ via the usual scalar product.

\smallskip

Motivated by the classical correspondence between %closed
geodesics in a Riemannian manifold  and %closed
characteristics of its unit cotangent bundle, %the following definition of 
we define $({\mathcal K},{\mathcal T})$-billiard trajectories  to be characteristics in % the unit cotangent bundle 
$U_{\mathcal  T}^*{\mathcal K}$ such that  their projections to ${\mathbb R}^n_q$ are  %would give  
%which are essentially 
closed billiard trajectories in ${\mathcal K}$ with a bouncing rule
that is determined by the geometry induced from the body ${\mathcal T}$; and vice versa, the projections to ${\mathbb R}^n_p$ are  %would give  
%which are essentially 
closed billiard trajectories in ${\mathcal T}$ with a bouncing rule
that is determined by %the geometry induced from 
%the body 
${\mathcal K}$.
More precisely, 
%, was given in~\cite{AAO1}.
%For a proper billiard trajectory, 
when we follow the  vector fields of the dynamics, % flow of the vector field ${\mathfrak X}$, 
we move in ${\mathcal K} \times \partial {\mathcal T}$ from $(q_0,p_0)$ to
$(q_1,p_0) \in \partial {\mathcal K} \times \partial {\mathcal T}$ following the inner normal to
$\partial {\mathcal T}$ at $p_0$. When we hit the boundary $\partial {\mathcal K}$ at the
point $q_1$, the vector field  changes, and we start to move in
$\partial {\mathcal K}  \times {\mathcal T}$ from $(q_1,p_0)$ to $(q_1,p_1) \in \partial {\mathcal K} \times \partial {\mathcal T}$ following the outer
 normal to $\partial {\mathcal K}$ at the point $q_1$.  Next, we move from
$(q_1,p_1)$ to $(q_2,p_1)$ following the opposite of the normal to
$\partial {\mathcal T}$ at $p_1$, and so on and so forth (see
Figure $1$). 
It is not hard to check that when one of the bodies, say ${\mathcal T}$, is a Euclidean
ball, then when considering the projection to ${\mathbb R}^{n}_q$, the bouncing rule described above is the
classical one (i.e., equal impact and
reflection angles). 
Hence, the above 
%Note that this 
reflection law is a
natural variation of the classical one %(i.e., equal impact and reflection angles) 
when the Euclidean structure on ${\mathbb R}^n_q$
is replaced by the metric induced by the norm $\| \cdot \|_{{\mathcal T}}$.
We continue with a more precise definition. 
%Moreover, it is not hard to check that when $T$ is the Euclidean
%unit ball, the bouncing rule described above is the
%classical one.
%Also, similarly to the Euclidean case, one can check that ${\mathcal T}$-billiard trajectories in ${\mathcal K}$   % $({\mathcal K},{\mathcal T })$-billiards 
%correspond to critical points of a length functional defined on the $j$-fold cross product of the boundary $\partial {\mathcal K}$, where the distances between two consecutive  points are measured with respect to  
% the support function $h_{\mathcal T}$, where $h_{\mathcal T}(u) = \sup \{ \langle x,u \rangle \, ; \, x \in {\mathcal T } \}$. 
%



\begin{definition} \label{def-of-periodic-traj} Given two smooth convex bodies ${\mathcal K} \subset {\mathbb R}^n_q$ and ${\mathcal T} \subset {\mathbb R}^n_p$. 
A closed $({\mathcal K},{\mathcal T})$-billiard trajectory is the image of a piecewise smooth
map $\gamma \colon S^1 \rightarrow \partial ({\mathcal K} \times {\mathcal T}) $
%such that % the following:
such that for every  $t \notin {\mathcal B}_{\gamma}:= \{ t
\in S^1 \, | \, \gamma(t) \in \partial {\mathcal K}  \times \partial {\mathcal T} \}$ one has
\begin{equation*}
\dot \gamma(t) = d \, {\mathfrak X}(\gamma(t)),  \end{equation*} for some positive
 constant $d$ and the vector field ${\mathfrak X}$  given by
\begin{equation*}
{\mathfrak X}(q,p) = \left\{
\begin{array}{ll}
(-\nabla g_{\mathcal T}(p) ,0), &   (q,p) \in int({\mathcal K}) \times \partial {\mathcal T},\\
(0,\nabla g_{\mathcal K}(q)), & (q,p) \in \partial {\mathcal K} \times int({\mathcal T}).
\end{array} \right.
\end{equation*}
Moreover, for any $t \in {\mathcal B}_{\gamma}$, the left and right
derivatives of $\gamma(t)$ exist, and
\begin{equation*} \label{eq-the-cone}
\dot \gamma^{\pm}(t) \in \{   \alpha (-\nabla g_{\mathcal T}(p) ,0) + \beta
(0,\nabla g_{\mathcal K}(q))    \ | \ \alpha,\beta \geq 0,  \ (\alpha, \beta) \neq (0,0) \}.
\end{equation*}
\end{definition}
% \begin{remark} {\rm 
Although  in Definition~\ref{def-of-periodic-traj} 
there is a natural symmetry between the  bodies ${\mathcal K}$ and ${\mathcal T}$, 
in what follows  we shall  assume that  ${\mathcal K}$ 
plays the role of the billiard table, while  ${\mathcal T}$ induces the geometry that governs the billiard dynamics in ${\mathcal K}$.
%It will be useful to introduce the 
We will use the following terminology: 
%Let $\pi_q \colon {\mathbb R}^{2n} \rightarrow {\mathbb R}^n_q$ denote the projection to the configuration space. 
for a $({\mathcal K},{\mathcal T})$-billiard trajectory $\gamma$, the curve $\pi_q(\gamma)$,  where $\pi_q \colon {\mathbb R}^{2n} \rightarrow {\mathbb R}^n_q$ is the projection of $\gamma$ to the configuration space, 
shall be called a {\it  ${\mathcal T}$-billiard trajectory in ${\mathcal K}$}. Moreover,  similarly to the Euclidean case, one can check that ${\mathcal T}$-billiard trajectories in ${\mathcal K}$   % $({\mathcal K},{\mathcal T })$-billiards 
correspond to critical points of a length functional defined on the $j$-fold cross product of the boundary $\partial {\mathcal K}$, where the distances between two consecutive  points are measured with respect to  
 the support function $h_{\mathcal T}$, where $h_{\mathcal T}(u) = \sup \{ \langle x,u \rangle \, ; \, x \in {\mathcal T } \}$.  %Moreover, there is a one-to-one correspondence between the action of 
%} \end{remark}

\begin{definition} %[{\bf Trajectories classification}]
A closed $({\mathcal K},{\mathcal T})$-billiard trajectory $\gamma$ is said to be {\it proper}
if the set ${\mathcal B}_{\gamma}$ is finite, i.e., $\gamma$ is a  
broken bicharacteristic that enters and instantly exits the boundary
$\partial {\mathcal K} \times \partial {\mathcal T}$ at the reflection points.
In the case where ${\mathcal B}_{\gamma} = S^1$, i.e., $\gamma$ is travelling
solely along the boundary $\partial {\mathcal K}  \times \partial {\mathcal T}$,
 we say that $\gamma$ is a {\it gliding trajectory}.
\end{definition}

\begin{figure} %[h1]
\begin{center}
\begin{tikzpicture}[scale=0.7]

 \draw[important line][rotate=30] (0,0) ellipse (75pt and 40pt);

 \path coordinate (w1) at (2.1,4*0.75) coordinate (q0) at
 (-4.5*0.5,-2*0.2) coordinate (q1) at (1.6,4*0.44) coordinate (q2) at
 (1.74,+0.46) coordinate (w2) at (2.6,-1.1) coordinate (w3) at (-3.2,-0.11) coordinate (K) at
 (0,0.15);

\draw[red] [important line] (q0) -- (q1); \draw[->] (q1) -- (w1);
\draw[red] [important line] (q1) -- (q2); %\draw[->] (q2) -- (w2);
\draw[->] (q0) -- (w3);

\filldraw [black]
  (w3) circle (0pt) node[above left=-2.5pt] {{\footnotesize $\nabla \|q_2\|_{\mathcal K}$}}
    (w1) circle (0pt) node[right] {{\footnotesize $\nabla \|q_1\|_{\mathcal K}$}}
    (w2) circle (0pt) % node[below ] {{\footnotesize $w_2=\nabla \|q_2\|_K$}}
     (q0) circle (2pt) node[below right] {{\footnotesize $q_2$}}
      (q1) circle (2pt) node[above right=0.5pt] {{\footnotesize $q_1$}}
       (q2) circle (2pt) node[below=0.5pt] {{\footnotesize $q_0$}}
        (K) circle (0pt) node[right=0.5pt] {${\mathcal K}$};

 %      % We start the second graph
       \begin{scope}[xshift=7cm]

 \draw[important line][rounded corners=10pt][rotate=10] (1.8,0) --
 (0.8,1.8)-- (-0.8,1.8)--  (-1.8,0)--  (-0.8,-1.8) -- (0.8,-1.8) --
 cycle;

 \path coordinate (p1) at (0.5,-4*0.433) coordinate (np0) at
 (2.3,2*0.9) coordinate (p0) at (1.2,2*0.53) coordinate (np1) at
 (0.7,-3) coordinate (p2) at (-1.2,2*0.66) coordinate (D) at
 (0.3,0.22);

 \draw[<-] (np0) node[right] {{\footnotesize $\nabla \|p_1
 \|_{\mathcal T}$}} -- (p0);
 \draw[<-] (np1) node[right] {{\footnotesize $\nabla \|p_0
 \|_{\mathcal T}$}} -- (p1);

 \draw[blue][important line] (p0) -- (p1);
 \draw[blue][important  line] (p0) -- (p2);

  \filldraw [black]
       (p1) circle (2pt) node[below right] {{\footnotesize $p_0$}}
         (p2) circle (2pt) node[left] {{\footnotesize $p_2$}}
         (p0) circle (2pt) node[above=2pt] {{\footnotesize $p_1$}}
          (D) circle (0pt) node[left] {${\mathcal T}$};
 \end{scope}
 \end{tikzpicture}

 \caption{A proper $({\mathcal K},{\mathcal T})$-Billiard trajectory.} 
 \end{center}
 \end{figure}


%For a proper billiard trajectory, when we follow the flow of the vector field
%${\mathfrak X}$, we move in ${\mathcal K} \times \partial {\mathcal T}$ from $(q_0,p_0)$ to
%$(q_1,p_0)$ following the inner normal to
%$\partial {\mathcal T}$ at $p_0$. When we hit the boundary $\partial {\mathcal K}$ at the
%point $q_1$, the vector field  changes, and we start to move in
%$\partial {\mathcal K}  \times {\mathcal T}$ from $(q_1,p_0)$ to $(q_1,p_1)$ following the outer
% normal to $\partial {\mathcal K}$ at the point $q_1$.  Next, we move from
%$(q_1,p_1)$ to $(q_2,p_1)$ following the opposite of the normal to
%$\partial {\mathcal T}$ at $p_1$, and so on  (see
%Figure $1$). 
%It is not hard to check that when ${\mathcal T}$ is a Euclidean
%ball, the bouncing rule described above is the
%classical one (i.e., equal impact and
%reflection angles). 
%Hence, the above 
%%Note that this 
%reflection law is a
%natural variation of the classical one %(i.e., equal impact and reflection angles) 
%when the Euclidean structure on ${\mathbb R}^n_q$
%is replaced by the metric induced by the norm $\| \cdot \|_{{\mathcal T}}$.
%%Moreover, it is not hard to check that when $T$ is the Euclidean
%%unit ball, the bouncing rule described above is the
%%classical one.
%Also, similarly to the Euclidean case, one can check that ${\mathcal T}$-billiard trajectories in ${\mathcal K}$   % $({\mathcal K},{\mathcal T })$-billiards 
%correspond to critical points of a length functional defined on the $j$-fold cross product of the boundary $\partial {\mathcal K}$, where the distances between two consecutive  points are measured with respect to  
% the support function $h_{\mathcal T}$, where $h_{\mathcal T}(u) = \sup \{ \langle x,u \rangle \, ; \, x \in {\mathcal T } \}$. 
%%
%%
%%Also, similarly to the Euclidean case, one can check that  $({\mathcal K},{\mathcal T })$-billiards 
%%correspond to critical points of   the length functional given by  the support function $h_{\mathcal T}$, where $h_{\mathcal T}(u) = \sup \{ \langle x,u \rangle \, ; \, x \in {\mathcal T } \}$. 
%%the reflection law is such that 
%%they are critical points for the length functional given by  the support function $h_T$, where $h_T(u) = \sup \{ \langle x,u \rangle \, ; \, x \in T \}$. 
%
%\smallskip
%
%We remark that i

The following theorem was proved in~\cite{AAO1}.
%In~\cite{AAO1} it was proved that 
%every $({\mathcal K},{\mathcal T})$-billiard trajectory is either a proper trajectory, or a gliding one, and that the following holds:
\begin{theorem} \label{Main-Theorem-From-AAO1}
Let ${\mathcal K}  \subset {\mathbb R}_q^n$, ${\mathcal T}  \subset {\mathbb R}_p^n$ be
two smooth  convex bodies.  Then, every $({\mathcal K},{\mathcal T})$-billiard trajectory is either a proper trajectory, or a gliding one.
Moreover, the Ekeland-Hofer-Zehnder capacity $c_{_{\rm EHZ}}({\mathcal K} \times {\mathcal T})$, of the Lagrangian product ${\mathcal K} \times {\mathcal T}$, is the length of the shortest periodic ${\mathcal T}$-billiard trajectory in ${\mathcal K}$, measured with respect to the support function $h_{\mathcal T}$. 
\end{theorem}

This theorem provides an effective way to estimate (and sometimes compute) the Ekeland-Hofer-Zehnder capacity of Lagrangian product configurations in the phase space. For example, in~\cite{AKO} (see Remark 4.2 therein) we used elementary tools from convex geometry to show that for centrally symmetric convex bodies, 
the shortest ${\mathcal T}$-billiard trajectory in ${\mathcal K}$ is a 2-periodic trajectory connecting a tangency point $q_0$ of ${\mathcal K}$ and a homotetic copy of $ {\mathcal T}^{\circ}$  to $-q_0$  (see Figure 2).  This result extends a previous result by Ghomi~\cite{Gh} for Euclidean billiards. 
In both cases, the main difficulty in the proof is to show 
%The main ingredient of the proof is to show t
that the above mentioned 2-periodic trajectory is indeed the shortest one. 
With this geometric observation at our disposal, we proved in~\cite{AKO}  the following result: denote by 
${\rm inrad}_{{\mathcal T}}({\mathcal K})  =  \max \{r \, | \, r{\mathcal T} \subset {\mathcal K} \}$. %in the case of the Hofer-
%Zehnder capacity) and estimate (for the other two capacities) large classes of convex domains
%in the phase space. It is very likely that the methods we developed in [3{7] can be imple-
%mented for more general congurations in the phase space. I plan to further pursue this
%line of research, aiming ultimately at proving Conjecture 1 above.


\begin{theorem} \label{Main-Theorem-From-AKO}
If ${\mathcal K}  \subset {\mathbb R}^n_q$, ${\mathcal T} \subset {\mathbb R}^n_p$ are centrally symmetric convex bodies, then
$$c_{_{\rm EHZ}}({\mathcal K}  \times {\mathcal T}) =  \overline c({\mathcal K}  \times {\mathcal T}) =  4 \, {\rm inrad}_{{\mathcal T}^{\circ}}({\mathcal K})$$%  = 4 \max \{r \, | \, r{\mathcal T}^{\circ} \subset {\mathcal K} \}. $$   
\end{theorem}
Note that Theorem~\ref{Main-Theorem-From-AKO} immediately implies Theorem~\ref{THM:CAP=4}  above, which in turn implies Theorem~\ref{THM-V-M}.
Thus, we have shown that Mahler's conjecture follows 
from a special case of Viterbo's conjecture.  %This result opens a promising new direction toward solving Mahler's long-standing conjecture by using symplectic methods, and 
In fact, it follows immediately from the proof of Theorem~\ref{THM-V-M}  that Mahler's conjecture is equivalent to Viterbo's conjecture when the latter is restricted to the Ekeland-Hofer-Zehnder capacity, and to convex domains of the form $\Sigma \times \Sigma^{\circ}$, where $\Sigma \subset {\mathbb R}^n_q$ is a centrally symmetric convex body.  We hope that further pursuing this line of research will lead to a  breakthrough in understanding both conjectures.


\begin{figure} 
\begin{center}
  \begin{tikzpicture}[scale=0.8] \label{Fig1}

\path coordinate (q1) at (-0.3,1.9) coordinate (q2) at (0.55,0.32) coordinate (q3) at (-0.3,1) coordinate (q4) at (0.3,-1) coordinate (q5) at (-0.80,1.9) coordinate (q6) at (0.8,-1.9)   ; 
\filldraw [black]
 (q3) circle (1pt) node[above] {{\footnotesize $\widetilde q$}}
 (q4) circle (1pt) node[below left=-0.7pt] {{\footnotesize $-\widetilde q$}}
 (q1) circle (0pt) node[below right=4pt] {{\footnotesize ${\mathcal K}$}}
(q2) circle (0pt) node[above right=-3.2pt] {{\footnotesize ${ r} \, {\mathcal T}^{\circ}$}};

\draw[dashed] (q3) -- (q4); 
 \draw[thick,->,blue] (q3)--(q5) node[left, black ] {${\scriptstyle  \nabla \| \widetilde q \|_{\mathcal K}}$};
  \draw[thick,->,blue] (q4)--(q6) node[right, black ] {${\scriptstyle \nabla \| - \widetilde q \|_{\mathcal K}}$};


  \draw[blue,rotate=30] (0,0) ellipse (2.2cm and 1cm);
  \draw[red,rotate=-80] (0,0) ellipse (1.05cm and 0.55cm);
      \begin{scope}[xshift=7cm]

\path coordinate (p1) at (-0.0,1.65) coordinate (p2) at (0.50,0.23) coordinate (p3) at (-0.75,1.32) coordinate (p4) at (0.75,-1.32) coordinate (p5) at (-0.7*1.4,1.2*1.8) coordinate (p6) at (0.7*1.4,-1.2*1.8)  ; 
\filldraw [black]
 (p2) circle (0pt) node[right=4pt] {{\footnotesize ${ r} \, {\mathcal K}^{\circ}$}}
(p1) circle (0pt) node[right] {{\footnotesize ${\mathcal T}$}};

\filldraw [black]
 (p3) circle (1pt) node[above right=-1pt] {{\footnotesize ${ } \widetilde p$}}
 (p4) circle (1pt) node[below left=-1pt] {{\footnotesize $-{ } \widetilde p$}};

\draw[dashed] (p3) -- (p4); 
 \draw[thick,->,red] (p3)--(p5) node[left, black ] {${\scriptstyle { \nabla \| \widetilde p \|_{\mathcal T}}}$};
  \draw[thick,->,red] (p4)--(p6) node[right, black ] {${\scriptstyle  \nabla \| - \widetilde p \|_{\mathcal T} }$};


  \draw[red,rotate=-20] (0,0) ellipse (1.8cm and 1.3cm);
  \draw[blue,rotate=-55] (0,0) ellipse (1.55cm and 0.7cm);
\end{scope}
\end{tikzpicture}

\caption{\it ${\mathcal T}$-billiard trajectory in ${\mathcal K}$ of length $4 \, {\rm inrad}_{{\mathcal T}^{\circ}}({\mathcal K})$.} 
\end{center}
\end{figure}

%\smallskip
\subsection{Bounds on the length of the shortest billiard trajectory}


Going somehow in the opposite direction,  %in our work~\cite{AO2} we used %the theory of 
one can also use  the theory of symplectic capacities to provide
several bounds and inequalities for the length of the shortest periodic
billiard trajectory in a smooth convex body in ${\mathbb R}^n$. 
In~\cite{AAO1} we prove the following theorem, which for the sake of simplicity we state only for the case of Euclidean billiards (for several other related results see~\cite{ABKS,AM,BB,Gh,IrirKei1, IrieKei2,V}). 
%We proved the following theorem, which for simplicity sake, we state in the case of classical Euclidean billiards. 
%
%We remark that 
%From the viewpoint of geometrical optics, Finsler billiards describe the propagation of waves in a nonhomogeneous,
%anisotropic medium containing %that contains 
%perfectly reflecting mirrors.
\begin{theorem} \label{THM-Billiard} Let  $K \subset {\mathbb R}^n$ be a smooth convex body, and let $\xi(K)$ denote the length of the shortest periodic billiard trajectory in $K$. Then, 
\begin{enumerate}
\item[(i)] $\xi(K_1) \leq \xi(K_2)$, for any convex domains $K_1 \subseteq K_2 \subseteq {\mathbb R}^n$ (monotonicity);
\item[(ii)] $\xi(K) \leq C \sqrt n \, {\rm Vol}(K)^{\frac 1 n},$ for some universal constant $C>0$;
\item[(iii)] $4  {\rm inrad}(K) \leq \xi(K) \leq 2 (n+1) {\rm inrad}(K)$;
\item[(iv)] $\xi(K_1 + K_2) \geq \xi(K_1) + \xi(K_2)$ (Brunn-Minkowski type inequality). 
\end{enumerate}
\end{theorem} 
We remark that the inequality $4 {\rm inrad}(K) \leq \xi(K)$ in ${\it (iii)}$ above was proved already in~\cite{Gh}, the monotonicity property was  
well known to experts in the field (although it has not been addressed in the literature to the best of our knowledge), and  all the results in Theorem~\ref{THM-Billiard} were later recovered and generalized by different methods (see~\cite{ABKS,IrirKei1, IrieKei2}).
Moreover, in light of the ``classical versus quantum" relation between the length spectrum in Riemannian geometry
and the Laplace spectrum, via trace formulae and Poisson relations, Theorem~\ref{THM-Billiard} %the theorem above 
can be viewed as
a classical counterpart of some  well-known results  for the first Laplace eigenvalue %  of the Dirichlet Laplace operator 
on convex domains. It is interesting to note that, to the best of the author's knowledge, the exact value of the constant $C$ in part  ${\it (ii)}$ of Theorem~\ref{THM-Billiard}   is unknown already in the two-dimensional case. 


\section{The Uniqueness of Hofer's Metric} \label{SEC:HOFER}

One of the most striking facts regarding % this group is that it carries an intrinsic geometry given by a Finsler bi-invariant metric known as Hofers metric.
%A remarkable fact, which is among the cornerstones of symplectic rigidity theory,
%$is that 
the group of Hamiltonian diffeomorphisms associated with  a symplectic manifold is that it can be
equipped with an intrinsic geometry given by a bi-invariant Finsler metric known as
Hofer's metric~\cite{H}. In contrast to the case of finite-dimensional Lie groups, the existence of such
a metric on an infinite-dimensional group of transformations is highly unusual due to
the lack of local  compactness.
Hofer's metric is exceptionally important for at least two reasons: first, Hofer showed in~\cite{H} that this metric 
%Among other things, 
%it 
gives rise to an important symplectic capacity known as  ``displacement energy", which turns out to have many different applications in symplectic topology and Hamiltonian dynamics (see e.g.,~\cite{Chek,H,HZ, L,LM,P,P1}). Second, it provides a certain geometric intuition for
the understanding of the long-time behaviour of Hamiltonian dynamical systems.

\smallskip

In~\cite{EliP}, Eliashberg and Polterovich initiated a discussion on the uniqueness of Hofer's metric
(cf.~\cite{Eli,P1}). They asked whether for a closed symplectic manifold $(M,\omega)$, Hofer's metric is the only 
%there exist a 
 bi-invariant Finsler metric on the group  of Hamiltonian diffeomorphisms. %which is not equivalent to Hofer's metric. 
In this section we explain  (following~\cite{OW} and~\cite{BO}) how tools from  classical functional analysis and the theory of normed function spaces  can be used to  positively  answer this question, and show that up to equivalence of metrics, Hofer's metric is unique. 
For this purpose, we now turn to more precise
formulations.

%In~\cite{OW} we have taken first steps toward answering this question, 
% In a joint work with L. Buhovskiy~\cite{BO}, we use tools 

 
%A natural question which was raised by Eliashberg--Polterovich in~\cite{EliP} is whether Hofer's metric is the only bi-invariant Finsler metric on the group of Hamiltonian diffeomorphism.

%
%In Ostrover-Wagner [43] we have taken first steps toward this
%direction. Among other things, we showed that for a closed symplectic manifold, the Finsler metric
%associated with a norm on the Lie algebra of Ham(M; !), which is bounded above by the L1-norm,
%vanishes identically.

\smallskip

Let $(M,\omega)$ be a closed $2n$-dimensional symplectic manifold,
and denote by $C^{\infty}_0(M)$ the space of smooth functions that
are zero-mean normalized with respect to the canonical volume form
$\omega^n$. With every smooth time-dependent Hamiltonian function $H: M \times [0,1] \rightarrow {\mathbb
R}$, %, traditionally called Hamiltonian function,
one associates a  vector field $X_{H_t}$ %on $M$
via the equation $i_{X_{H_t}} \omega = - dH_t$, where $ H_t(x) =
H(t,x)$. The flow
of  $X_{H_t}$ %, which is called the Hamiltonian flow of $H$,
is denoted by $\phi_H^t$ and is defined for all $t \in [0,1]$.
The %%main object of this note is the 
group of Hamiltonian
diffeomorphisms consists of all the time-one maps of such
Hamiltonian flows, i.e.,
$$ {\rm Ham}(M,\omega) = \{ \phi_H^1 \ |  \ \phi_H^t \
{\rm is \ a \ Hamiltonian \ flow  } \}.$$ When  equipped with the
standard $C^{\infty}$-topology, the group 
${\rm Ham}(M,\omega)$ %of Hamiltonian diffeomorphisms
is an infinite-dimensional Fr\'echet Lie group. Its Lie algebra, denoted here by 
${\cal A}$, can be naturally identified with the space of normalized smooth functions $C^{\infty}_0(M)$.
Moreover, the adjoint action of Ham$(M,\omega)$ on ${\cal A}$ is the
standard action of diffeomorphisms on functions, i.e., ${\rm Ad}_\phi f = f
\circ \phi^{-1}$, for every $f \in {\cal A}$  and $\phi \in$
Ham$(M,\omega)$. For more details on the group of  Hamiltonian
diffeomorphisms see e.g.,~\cite{HZ, McSal,P1}. 

\smallskip

Next, we define a Finsler pseudo-distance on
Ham$(M,\omega)$. Given any pseudo-norm $\| \cdot \|$ on % the Lie algebra 
${\cal A}$, we define the length of a path $\alpha : [0,1]
\rightarrow {\rm Ham}(M,\omega)$ as
$$ {\rm length}\{ {\alpha}\} = \int_0^1 \| \dot \alpha \| dt =
\int_0^1 \| H_t \| dt ,$$ where $H_t(x)=H(t,x)$ is the unique
normalized Hamiltonian function generating the path $\alpha$. Here
$H$ is said to be normalized if $\int_M H_t \omega^n=0$ for every
$t\in [0,1]$. The distance between two Hamiltonian diffeomorphisms
is given by $$ d(\psi,\varphi) := \inf {\rm length} { \{ \alpha \}
},$$ where the infimum is taken over all Hamiltonian paths $\alpha$
connecting $\psi$ and $\varphi$. It is not hard to check that $d$ is
non-negative, symmetric, and satisfies the triangle inequality.
Moreover, any pseudo-norm on the Lie algebra ${\cal A}$ that is invariant under the
adjoint action yields a bi-invariant pseudo-distance function on ${\rm Ham} (M,\omega)$, i.e.,
 $d(\psi,\phi) = d(\theta \, \psi,\theta \, \phi) = d(\psi \, \theta ,\phi \, \theta)$,
 for every  $\psi,  \phi,  \theta \in {\rm Ham} (M,\omega)$.


\smallskip


{\bf From here forth we  deal solely with such pseudo-norms
and we  refer to $d$
as the pseudo-distance generated by the pseudo-norm $\| \cdot \|$. }


\smallskip


We remark in passing that
%\begin{remark} \label{Rmk-about-continuity} {\rm
a fruitful study of right-invariant Finsler metrics on
Ham$(M,\omega)$, motivated in part by applications to hydrodynamics,
was initiated  by Arnold~\cite{Ar}. % (cf.~\cite{AK} and the references within). 
In addition, non-Finslerian bi-invariant metrics on Ham$(M,\omega)$ have been
intensively studied in the realm of symplectic geometry, starting
with the works of Viterbo~\cite{V1}, Schwarz~\cite{Sch}, and
Oh~\cite{Oh}, and followed by many others. %} \end{remark} 
\begin{remark} \label{Rmk-about-continuity} {\rm
When one studies  geometric properties of the group of
Hamiltonian diffeomorphisms, it is convenient to consider smooth
paths $ [0,1] \rightarrow {\rm Ham}(M,\omega) $, among which those
that start at the identity correspond to smooth Hamiltonian flows.
Moreover, for a given Finsler pseudo-metric on $ {\rm Ham}(M,\omega)$, a natural geometric assumption is that every
smooth path $ [0,1] \rightarrow {\rm Ham}(M,\omega) $ has finite
length. As it turns out,  the latter 
assumption is equivalent to the
continuity of the pseudo-norm on ${\cal A}$ corresponding to the
pseudo-Finsler metric in the $ C^{\infty} $-topology (see~\cite{BO}).
%\footnote{We
%thank A. Katok for his illuminating remark regarding the naturalness
%of the assumption that the pseudo norm is continuous in the $
%C^\infty $-topology.}.
% this fact.}.
%We prove this fact in the Appendix to the paper. 
Thus, in what follows %hroughout the text
we shall mainly consider such pseudo-norms.}
 \end{remark}

It is highly non-trivial to check whether a distance function on the group of Hamiltonian diffeomorphisms 
generated by  a pseudo-norm is non-degenerate, that is, $d({\rm Id},\phi)
> 0$ for $\phi \neq {\rm Id}$. In fact, for closed
symplectic manifolds, a bi-invariant pseudo-metric $d$ on
Ham$(M,\omega)$ is either a genuine metric or identically zero. This
is an immediate corollary of a well-known theorem by
Banyaga~\cite{B}, which states that Ham$(M,\omega)$ is a simple
group, combined with the fact that the null-set $${\rm null}(d) = \{
\phi \in {\rm Ham}(M,\omega) \ | \ d({\rm Id},\phi) = 0 \}$$ is a normal
subgroup of Ham$(M,\omega)$. A  renowned result by
Hofer~\cite{H} states that the $L_{\infty}$-norm on ${\cal A}$ gives
rise to a genuine distance function on Ham$(M,\omega)$ known now as
Hofer's metric. This was  proved by Hofer for the case
of ${\mathbb R}^{2n}$, then generalized by Polterovich~\cite{P}, and
finally proven in full generality by Lalonde and McDuff~\cite{LM}.
In a sharp contrast to the above, Eliashberg and
Polterovich showed in~\cite{EliP} that for a closed symplectic manifold $(M,\omega)$ ons has
\begin{theorem}[Eliashberg and Polterovich]
For $1 \leq p < \infty$,  the
pseudo-distances on ${\rm Ham}(M,\omega)$ corresponding to the
$L_p$-norms on ${\cal A}$ vanish identically.
\end{theorem}
% that for $1 \leq p < \infty$,  the
%pseudo-distances on ${\rm Ham}(M,\omega)$ corresponding to the
%$L_p$-norms on ${\cal A}$ vanishes identically. 

The following question was asked in~\cite{EliP} (cf.~\cite{Eli,P1}):

\begin{question} \label{Ques:inv-norms} What are the ${\rm Ham}(M,\omega)$-invariant norms on ${\cal
A}$, and which of them give rise to genuine bi-invariant metrics
on ${\rm Ham}(M,\omega)$?
\end{question}

It was observed in~\cite{BO} that  any pseudo-norm $\| \cdot \|$ 
on the space ${\mathcal A}$ can be turned into a Ham$(M,\omega)$-invariant pseudo-norm via a certain 
invariantization procedure $ \| f \| \mapsto \| f \|_{\rm inv}$. 
The idea behind this procedure is based on the notion of infimal convolution (or epi-sum), from convex analysis. 
Recall that the infimal convolution of two functions $f$ and $g$ on ${\mathbb R}^n$ is defined by $(f \square g)(z)  = \inf \{ f(x)+g(y) \, | \, z=x+y\}$. 
This operator has a simple geometric interpretation: the epigraph (i.e., the set of points lying on or above the graph) of the infimal convolution of two functions is the Minkowski sum of the epigraphs of those functions. 
The invariantization $\| \cdot \|_{\rm inv}$  of $\| \cdot \|$ is obtained by taking the orbit of $\|  \cdot \|$ under the group action, and consider the infimal convolution of the associated family of norms. More preciesly, define


%Let us emphasize that any pseudo-norm %$\| \cdot \|$ 
%on ${\mathcal A}$
%can be turned into a Ham$(M,\omega)$-invariant pseudo-norm via the
%invariantization procedure $ \| f \| \mapsto \| f \|_{\rm inv}$, where:
$$ \| f \|_{\rm inv} =  \inf \Bigl \{  \sum \|\phi_i^*  f_i \|  \ ; \ f = \sum f_i, \ {\rm and \ } \phi_i \in {\rm Ham}(M,\omega)  \Bigr \}. $$
We remark that in the above definition of $\| f \|_{\rm inv}$ the sum
$\sum f_i$ is assumed to be finite. Note that $\| \cdot \|_{\rm inv}
\leq \| \cdot \|$.  Thus, if $\| \cdot \|$ is continuous in the
$C^{\infty}$-topology, then so is $\| \cdot \|_{\rm inv}$. Moreover, if
$\| \cdot \|'$ is a Ham$(M,\omega)$-invariant pseudo-norm, then:
$$ \| \cdot \|' \leq \| \cdot \| \Longrightarrow \| \cdot \|' \leq \| \cdot \|_{\rm inv}.$$
In particular, the above invariantization procedure provides a
plethora of Ham$(M,\omega)$-invariant genuine norms on ${\mathcal
A}$, e.g., by applying it to the $\| \cdot \|_{C^k}$-norms.
%the above invariantization procedure on the $\| \cdot \|_{C^k}$-norms.
%taking the homogenization of the $\| \cdot \|_{C^k}$-norms. }
% }\end{remark}


\smallskip

In~\cite{OW} we made a first step toward answering Question~\ref{Ques:inv-norms} using tools from the theory of normed spaces and functional analysis. More precisely, regarding the first part of Question~\ref{Ques:inv-norms}, we proved
%Our main contributions towards answering Question~\ref{Ques:inv-norms} are



\begin{theorem}[Ostrover and Wagner] \label{Ham-invariant-implies-measure-invarinat}
Let $ \| \cdot \|$ be a ${\rm Ham}(M,\omega)$-invariant norm on ${\cal
A}$ such that $\| \cdot \| \leq C \| \cdot \|_{\infty}$ for some
constant $C$. Then $\| \cdot \|$ is invariant under all measure
preserving diffeomorphisms of $M$.
\end{theorem}
In other words, any ${\rm Ham}(M,\omega)$-invariant  norm on  ${\cal A}$ that is bounded above by the $L_{\infty}$-norm, must also be invariant  under the much larger group of measure preserving diffeomorphisms. 
%\noindent Here, two norms are said to be \emph{equivalent}, if
%each bounds the other up to a multiplicative constant.
%\noindent The next result is a strengthened formulation of
%Theorem~\ref{Ham-invariant-implies-measure-invarinat} and a key
%ingredient in the proof of Theorem~\ref{The_main_theorem}. As the
%discussion below explains, it also bears on the question of
%classifying Ham$(M,\omega)$-invariant norms.
%
%\begin{theorem}\label{extension-to-L-infinity}
%Let $ \| \cdot \|$ be a Ham$(M,\omega)$-invariant norm on ${\cal
%A}$ such that $\| \cdot \| \leq C \| \cdot \|_{\infty}$ for some
%constant $C$. Then $\| \cdot \|$ can be extended to a semi-norm
%$||| \cdot ||| \leq C\| \cdot \|_{\infty}$ on $L_{\infty}(M)$,
%which is invariant under all measure preserving bijections on $M$.
%\end{theorem}
%
%A considerable
%generalization of the latter result was given by
%Ostrover-Wagner~\cite{OW} who proved that for a closed symplectic
%manifold:
Theorem~\ref{Ham-invariant-implies-measure-invarinat} plays an important role in the proof of the following result, which gives a partial answer to the second part of  Question~\ref{Ques:inv-norms}.
\begin{theorem}[Ostrover and Wagner] \label{OW-theorem}
Let $ \| \cdot \|$ be a ${\rm Ham}(M,\omega)$-invariant norm on ${\cal A}$
such that $\| \cdot \| \leq C\| \cdot \|_{\infty}$ for some constant
$C$, but the two norms are not equivalent.\footnote{Two norms are said to be equivalent
 if  ${\frac 1 C} \, \| \cdot \|_1 \leqslant \| \cdot \|_2 \leqslant C \| \cdot \|_1$ for some constant $C>0$.}
%
%exist two constants such that }. 
Then the associated
pseudo-distance $d$ on ${\rm  Ham}(M,\omega)$ vanishes identically.
\end{theorem}
Although Theorem~\ref{OW-theorem} gives a partial answer to the second part of  Question~\ref{Ques:inv-norms}, prima facie, there might be ${\rm  Ham}(M,\omega)$-invariant norms on ${\cal A}$  which  are either strictly bigger than the $L_{\infty}$-norm, or  incomparable to it.
In a joint work with L. Buhovsky~\cite{BO} we showed that under the natural continuity assumption mentioned in Remark~\ref{Rmk-about-continuity} above, this cannot happen. Hence, up to equivalence of metrics,  Hofer's metric is unique. More precisely, 
%
%
%In~\cite{BO} we provide a complete answer to the uniqueness question of Hofer's metric % answer to the above question %for closed symplectic manifolds
%under the natural continuity assumption mentioned in
%Remark~\ref{Rmk-about-continuity}. %More precisely, 
%

\begin{theorem}[Buhovsky and Ostrover]  \label{Main-thm-BO} Let $(M,\omega)$ be a closed symplectic manifold.
Any $C^{\infty}$-continuous {\rm Ham}$(M,\omega)$-invariant pseudo-norm $\| \cdot \|$ on
${\mathcal A}$
%Any {\rm Ham}$(M,\omega)$-invariant pseudo norm $\| \cdot \|$ on
%${\mathcal A}$ that is continuous in the $C^{\infty}$-topology, 
is dominated  from above by the $L_{\infty}$-norm i.e., $\| \cdot \|
\leq C \| \cdot \|_{\infty}$ for some constant $C$.
\end{theorem}


Combining Theorem~\ref{Main-thm-BO} and Theorem~\ref{OW-theorem} above, we obtain:


\begin{corollary} For a closed symplectic manifold $(M,\omega)$, any bi-invariant Finsler pseudo-metric on ${\rm Ham}(M,\omega)$,
obtained by a pseudo-norm $\| \cdot \|$ on ${\mathcal A}$ that is
continuous in the $C^{\infty}$-topology, is either identically zero, 
or equivalent %t\footnote{Two metrics $d_1,d_2$ are said to be
%equivalent if  ${\frac 1 C} \, d_1 \leqslant d_2 \leqslant C d_1$
%for some constant $C>0$.} 
to Hofer's metric. In particular, any
non-degenerate bi-invariant Finsler metric on ${\rm Ham}(M,\omega)$ which
is generated by a norm that is continuous in the
$C^{\infty}$-topology  gives rise to the same topology on
${\rm Ham}(M,\omega)$ as the one induced by Hofer's metric.
%coming from the Hofer metric
\end{corollary}



%\subsection{Outline of the Proof} \label{section-outline}

%Here 
In the rest of this section we briefly describe the strategy of the proof of
Theorem~\ref{Main-thm-BO} in the two-dimensional case. For the proof of the general case see~\cite{BO}. 
We start with two straightforward reduction steps. 
First, for technical reasons, 
%
%
%The first is that 
%for technical reasons, we shall discuss  the proof of 
%Theorem~\ref{Main-thm-BO} for 
%
we shall consider 
pseudo-norms on the space
$C^{\infty}(M)$,  instead of the space ${\mathcal A}$.
The original claim will follow, %from this result 
since any
Ham$(M,\omega)$ invariant pseudo-norm $\| \cdot \|$ on ${\mathcal
A}$ can be naturally extended to an invariant pseudo-norm $\| \cdot
\|'$ on $C^{\infty}(M)$ by % setting
$$\| f \|' := \| f- M_f \|, \ {\rm where \ } M_f = {\textstyle {\frac 1 {\rm Vol(M)}} \int_M f \omega^n}.$$ Note that if
$\| \cdot \|$ is continuous in the $C^{\infty}$-topology, then so is
$\| \cdot \|'$, and that the two norms coincide 
%Moreover, the pseudo norm $\| \cdot \|'$ coincides
%with $\| \cdot \|$ 
on the space ${\mathcal A}$. Second, by using a standard
partition of unity argument, we can reduce the proof of Theorem~\ref{Main-thm-BO}
to a ``local result", i.e., %we show that 
it is sufficient to prove the
theorem for Ham$_c(W,\omega)$-invariant pseudo-norms
on the space of compactly supported smooth functions $C_c^{\infty}(W)$, where $W=(-L,L)^{2}$ is an open %$2n$-dimensional
square in ${\mathbb R}^{2}$ (see~\cite{BO} for the details).

\smallskip

The next step, which is one of the key ideas of the proof, is to define the ``largest possible"  Ham$_c(W,\omega)$-invariant norm on the space of compactly supported smooth functions $C_c^{\infty}(W)$. To this end, 
we fix a (non-empty) finite collection of functions ${\mathcal F} \subset C_c^{\infty}(W)$, and define: 
%
%As a first step toward this end, % the proof of this local version of the theorem,
%we introduce a special Ham$_c(W,\omega)$-invariant norm $\| \cdot
%\|_{{\mathcal F}, max}$ on $C_c^{\infty}(W)$, which depends on a
%given finite collection ${\mathcal F} \subset C_c^{\infty}(W)$. More
%precisely:
%
%%\begin{mydef1} 
%\begin{definition}
%For a non-empty finite collection ${\mathcal F} \subset C^{\infty}_c(W)$, let
\begin{multline*}
 {\cal L}_{\mathcal F} := \Bigl \{ \sum_{i,k} c_{i,k} \, \Phi_{i,k}^* {f}_i \ | \ c_{i,k} \in {\mathbb R},
\ \Phi_{i,k} \in {\rm Ham}_c(W,\omega),  \\  \ {f}_i \in {\mathcal F},\  
{\rm and} \ \# \{(i,k) \, | \, c_{i,k} \neq 0 \} < \infty \Bigr  \}.
\end{multline*}
%
%
%$$  {\cal L}_{\mathcal F} := \Bigl \{ \sum_{i,k} c_{i,k} \, \Phi_{i,k}^* {f}_i \ | \ c_{i,k} \in {\mathbb R},
%\ \Phi_{i,k} \in {\rm Ham}_c(W,\omega), \ {f}_i \in {\mathcal F},\ 
%{\rm and} \ \# \{(i,k) \, | \, c_{i,k} \neq 0 \} < \infty \Bigr  \},
%$$
%
%
We equip  the space  ${\cal L}_{\mathcal F} $ with the norm $$ \| f \|_{{\cal L}_{{\mathcal F}}} = \inf
\sum |c_{i,k}|,$$ where the infimum is taken over all the
representations $f = \sum c_{i,k} \, \Phi_{i,k}^* {f}_i$ as above.
%\end{mydef1}
%\end{definition}
\begin{definition}
%\begin{mydef2} %\label{definition1-of-our-max-norm-local}
For any compactly supported function $ f \in C_c^{\infty}(W) $, let
\begin{equation*} % \label{definition-of-max-norm-local}
 \| f \|_{{\cal F}, \, {\rm max}} = \inf \big\{ \liminf_{i \rightarrow \infty} \| f_i
\|_{{\mathcal L}_{{\mathcal F}}} \big\} ,\end{equation*} where the infimum is
taken over all subsequences $\{f_i\}$ in $ {\cal L}_{\mathcal F} $ which
converge to $f$ in the $C^{\infty}$-topology. As usual, the infimum
of the empty set is set to be $+ \infty$.
%If such sequence do not exists, we set $ \| f \|_{{\cal F}, \, max} \equiv \infty$.
\end{definition}
%\end{mydef2}

The main feature of the norm $\| \cdot \|_{{\cal F}, \, {\rm max}}$ is %the fact
that it dominates from above any other Ham$_c(W,\omega)$-invariant
pseudo-norm that is continuous in the
$C^{\infty}$-topology.  %\footnote{It is a priori unclear whether the
%norm $\| \cdot \|_{{\mathcal F}, max}$ itself is continuous in the
%$C^{\infty}$-topology. This would follow subsequently from
%Proposition~\ref{Ck-bound-lemma} part (ii) ({\bf Lev: this is true
%only for special} ${\mathcal F}$.})
%More preciesly,
\begin{lemma} \label{lemma-about-max-norm}
Let ${\mathcal F} \subset C_c^{\infty}(W)$ be a non-empty finite
collection of smooth compactly supported functions in $W$. Then  any
{\rm Ham}$_c(W,\omega)$-invariant pseudo-norm $ \| \cdot \| $ on $
C_c^\infty(W) $ that is continuous in the $ C^\infty $-topology
satisfies $$ \| \cdot \| \leqslant C \| \cdot \|_{{\cal F}, \, {\rm max}},$$
for some absolute constant $C$.
\end{lemma}

\begin{proof}[\bf Proof of Lemma~\ref{lemma-about-max-norm}]
Since the collection ${\mathcal F}$ is finite, set $C = \max \{ \| g \|  ; \, g \in {\mathcal F} \}$. For any $f =
\sum c_{i,k} \, \Phi_{i,k}^*  f_i \in {\cal L}_{\mathcal F}$, one
has
\begin{equation} \label{simpel-estimate1}  \|f \| \leq   \sum |c_{i,k}| \| \Phi_{i,k}^* f_i \|  \leq  C \sum |c_{i,k}|. %    \leq C  \| f \|_{{\mathcal L}_{{\mathcal F}}} %   _{{\cal F}, \,max}
\end{equation}
By the definition of $\| \cdot \|_{{\mathcal L}_{{\mathcal F}}}$, this immediately implies that $\|f\| \leq  C  \| f \|_{{\mathcal L}_{{\mathcal F}}}$.
The lemma now follows by combining~$(\ref{simpel-estimate1})$, the
definition of  $ \| \cdot \|_{{\cal F}, \, {\rm max}} $, and the fact that
the pseudo-norm $ \| \cdot \| $ is assumed to be continuous in the $
C^\infty $-topology.
\end{proof}


\smallskip

The next step, which is the
main part of the proof, is to show that for a suitable collection of
functions ${\mathcal F} \subset C_c^{\infty}(W)$, the norm $\| \cdot
\|_{{\mathcal F}, \, {\rm max}}$ is in turn bounded from above by the
$L_{\infty}$-norm. % i.e., any  $f \in C_c^{\infty}(W^2)$ satisfies $\| f
%\|_{{\mathcal F}, \, max} \leqslant C \| f \|_{{\infty}}$ for some
%absolute constant $C$. 
 %This is proved in Theorem~\ref{Main-Thm-local-case}, and 
In light of the above, this would complete the proof of Theorem~\ref{Main-thm-BO} in the two-dimensional case. 
%The proof of
%Theorem~\ref{Main-Thm-local-case} is divided into two main steps
%which we now turn to describe:
%
%
%
%
%
%\noindent  {\bf The local two-dimensional case:} Here, we shall
%construct a collection ${\mathcal F}$ of smooth compactly supported
%functions on a two-dimensional cube $W^2 \subset {\mathbb R}^{2}$,
%such that any  $f \in C_c^{\infty}(W^2)$ satisfies $\| f
%\|_{{\mathcal F}, \, max} \leqslant C \| f \|_{{\infty}}$ for some
%absolute constant $C$. 


\smallskip



There are two independent components  in the
proof of this claim. First, we show that one can
decompose any $f \in C_c^{\infty}(W^2)$ with $\| f \|_{{\infty}} \leqslant 1$
into a finite combination $f = \sum_{i=1}^{N_0} \epsilon_j \Psi^*_j g_j$. Here, $ \epsilon_j \in \{ -1, 1 \} $, $\Psi_j \in {\rm Ham}_c(W^2,\omega)$, and $g_j$ are smooth rotation-invariant %Guassian-like
functions  whose $L_{\infty}$-norm is bounded by an absolute
constant, and
%
%which are uniformly bounded in the $\| \cdot \|_{L_{\infty}}$-norm, and
which satisfy certain other technical conditions (see
Proposition 3.5 in~\cite{BO} for the precise
statement). In what follows we call such functions ``simple
functions". We emphasize that $N_0$ is a constant independent of
$f$. Thus, we can restrict ourselves to the case where $f$ is a
``simple function''. In the second part of the proof, we construct
an explicit collection ${\mathcal F} = \{ {\mathfrak f_0},
{\mathfrak f_1} , {\mathfrak f_2} \} $, where ${\mathfrak f_i} \in
C_c^{\infty}(W^2),$ and $i=0,1,2$. %which consist of three functions.
Using an averaging procedure
(see the proof of Theorem 3.4 in~\cite{BO}), one can show that every
``simple function'' $f \in C_c^{\infty}(W^2)$ can be approximated
arbitrarily well in the $C^{\infty}$-topology by a sum of the form
%$$ \sum \alpha_{i_j} \widetilde \Psi_{i_j}^* {\mathfrak f}_J \rightarrow g_j$$
$$ \sum_{i,k} \alpha_{i,k} \widetilde \Psi_{i,k}^* {\mathfrak f}_{k}, \ {\rm where \ }  \widetilde \Psi_{i,k} \in {\rm Ham}_c(W^2,\omega), \ k \in \{0,1,2 \}, $$ %\rightarrow f$$
and such that $\sum | \alpha_{i,k} | \leq C \| f \|_{\infty}$ for
some absolute constant $C$. Combining this with the above definition
of $\| \cdot \|_{{\mathcal F}, \, {\rm max}}$, we conclude that
%This implies that
$ \| f \|_{{\mathcal F}, \, {\rm max}} \leq C \| f \|_{\infty}$ for every
$f \in C_c^{\infty}(W^2)$. Together with Lemma~\ref{lemma-about-max-norm}, this completes the proof of
Theorem~\ref{Main-thm-BO} in the 2-dimensional case.


%\noindent  {\bf The local higher-dimensional case:} The proof of
%Theorem~\ref{Main-Thm-local-case} for arbitrary dimension strongly
%relies on the 2-dimensional case. We extend (in a natural way) the
%construction of the above mentioned collection ${\mathcal F} = \{
%{\mathfrak f_0},{\mathfrak
%f_1},{\mathfrak f_2} \}$  %from the $2$-dimensional case
%to the $2n$-dimensional case. By abuse of notation, we shall denote
%the new collection by ${\mathcal F}$ as well. Based on the proof of
%Theorem~\ref{Main-Thm-local-case} in the $2$-dimensional case, and
%on the construction of the class ${\mathcal F}$, we show that
%Theorem~\ref{Main-Thm-local-case} holds for ``product functions'',
%i.e., for $f \in C_c^{\infty}(W)$ of the form $f = \prod_{i=1}^n
%f_i(q_i,p_i)$, where $f_i \in C_c^{\infty}(W^2)$.
%
%%This is a direct consequence from the proof in the two-dimensional case and the definition
%
%%of $\| \cdot \|_{{\mathcal F}, max}$.
%
%From this we derive, using a Fourier series argument, that the norm
%$\| \cdot \|_{{\mathcal F}, max}$ is dominated from above by the $\|
%\cdot \|_{C^{2n+1}}$-norm, i.e., for any $f \in C_c^{\infty}(W)$ one
%has \begin{equation} \label{eq-in-outline-about-C2n-norm}  \|f
%\|_{{\mathcal F}, \, max} \leq C \|f \|_{C^{2n+1}}, \end{equation}
%for some constant $C$  (see Proposition~\ref{Ck-bound-lemma} for the
%proof of the above two claims).
%
%% \begin{figure} %[h1]
%
%% \begin{center}
%
%%   \begin{tikzpicture}[scale=0.8]
%
%% \draw[step=1.15cm,color=gray]
%
%% (0,0) grid (3.45,3.45);
%
%% \draw [important line] plot coordinates {
%
%% (-6/10+0.57,0.1*0.5+0.35) (-1/2+0.57,0.1*0.5+0.35) (-0.9/2+0.57,0.1003*0.5+0.35)  (-0.8/2+0.57,0.11914*0.5+0.35)  (-0.7/2+0.57,0.1933*0.5+0.35)  (-0.6/2+0.57,0.3401*0.5+0.35) (-0.5/2+0.57,0.5474*0.5+0.35) (-0.4/2+0.57,0.7758*0.5+0.35) (-0.3/2+0.57,0.9719*0.5+0.35) (-0.2/2+0.57,1.0859*0.5+0.35)  (-0.1/2+0.57,1.1*0.5+0.35) (0+0.57,1.1*0.5+0.35) (0.1/2+0.57,1.1*0.5+0.35) (0.2/2+0.57,1.0859*0.5+0.35) (0.3/2+0.57,0.9719*0.5+0.35) (0.4/2+0.57,0.7758*0.5+0.35) (0.5/2+0.57,0.5474*0.5+0.35)
%
%% (0.6/2+0.57,0.3401*0.5+0.35) (0.7/2+0.57,0.1933*0.5+0.35) (0.8/2+0.57,0.11914*0.5+0.35) (0.9/2+0.57,0.1003*0.5+0.35) (1/2+0.57,0.1*0.5+0.35) (6/10+0.57,0.1*0.5+0.35)};
%
%%       \begin{scope}[xshift=4cm]
%
%% \draw[step=1.15cm,color=gray]
%
%% (0,0) grid (3.45,3.45);
%
%% \begin{scope}[xshift=4cm]
%
%% \draw[step=1.15cm,color=gray]
%
%% (0,0) grid (3.45,3.45);
%
%% \begin{scope}[xshift=4cm]
%
%% \draw[step=1.15cm,color=gray]
%
%% (0,0) grid (3.45,3.45);
%
%% \end{scope}
%
%% \end{scope}
%
%% \end{scope}
%
%% \end{tikzpicture}
%
%% \caption{BLA}
%
%% \end{center}
%
%% \end{figure}
%
%Next, for any $\epsilon > 0$, we construct a partition of unity
%function ${\mathcal R}^{\epsilon}: \mathbb{R}^{2n} \rightarrow
%\mathbb{R} $,
%with $ supp( {\mathcal R}^{\epsilon} ) \subset (-\epsilon,\epsilon)^{2n} $, and such that %such that one has a partition of unity
%$$ \sum_{v \in \epsilon \mathbb{Z}^{2n}}  {\mathcal R}^{\epsilon}(x-v) = {\rm Id}(x) $$
%For any $w \in { \mathfrak X}:= \{ 0,1,2,3 \}^{2n}$, we consider a finite grid  $\Gamma^{\epsilon}_{w} \subset W$ given by: %,let %we denote by
%$$ \Gamma^{\epsilon}_{w} =
%\epsilon w + 4 \epsilon \mathbb{Z}^{2n} \cap (-L+
%3\epsilon,L-3\epsilon)^{2n}, $$ %be a finite grid in $W$,
%and define
%$$ f_{w}(x) = \sum_{v \in \Gamma^{\epsilon}_w} {\cal
%R}^{\epsilon}(x-v) f(x) $$
%Note, that for $ \epsilon $ sufficiently small %enough
%such that $ supp \, (f) \subset (-L+4\epsilon,L-4\epsilon)^{2n} $,
%one has $$ f(x) = \sum_{w \in {\mathfrak X}} f_{w}(x) $$ For any $ w
%\in {\mathfrak X} $, the function $ f_{w} $ is a finite
%sum of smooth functions %with ``small" and mutually disjoint supports
%that lie near the points of the grid $ \Gamma^{\epsilon}_{w}$.
%Moreover, these functions have mutually disjoint supports, which are
%spaced commodiously. % (see picture 1).
%Next, we fix $ w \in {\mathfrak X} $, and for any $ v \in
%\Gamma^{\epsilon}_w $ we consider the decomposition of $f \in
%C^{\infty}_c(W)$ as a Taylor polynomial of order $2n+1$ and a
%remainder, around the point $ v $ (this specific choice of the order
%ensures, based on~$(\ref{eq-in-outline-about-C2n-norm})$, the
%estimate~$(\ref{eq-bound-for-h_w})$ below):
% $$ f(x) = P_{2n+1}^v(x-v) +
%R_{2n+1}^v(x-v).$$ We decompose each $f_w$ as $ f_w(x) = g_w(x) +
%h_w(x) $, where
%$$ g_{w}(x) = \sum_{v \in \Gamma^{\epsilon}_w} {\cal
%R}^{\epsilon}(x-v) P_{2n+1}^v(x-v) , \ {\rm and}  \ \ h_{w}(x) =
%\sum_{v \in \Gamma^{\epsilon}_w} {\cal R}^{\epsilon}(x-v)
%R_{2n+1}^v(x-v) .$$ Based on~($\ref{eq-in-outline-about-C2n-norm}$),
%in Lemma~\ref{lemma-C^k-estimate-of-the-reminder} (cf.
%Corrolary~\ref{cor-about-max-norm-of-the-reminder}) we show that the
%$\| \cdot \|_{{\mathcal F}, max}$-norm of the remainder parts $\{
%h_w \}$ can be taken to be arbitrarily small. More precisely,
%
%\begin{equation} \label{eq-bound-for-h_w}
% \| h_{w} \|_{{\mathcal F}, max} \leqslant C_1 \| h_{w} \|_{C^{2n+1}} \leqslant C_2 \epsilon
%\|f\|_{C^{2n+2}} ,
%\end{equation}
% for some constants $C_1$ and $C_2$.
%On the other hand, using a combinatorial argument and the above
%mentioned fact that Theorem~\ref{Main-Thm-local-case} holds for
%``product functions", we prove the estimate
%\begin{equation} \label{eq-bound-for-g_w}
% \| g_{w} \|_{{\mathcal F}, \, max} \leqslant C_3 \bigl ( \sum_{i=0}^{2n+1}
%\|f\|_{C^i} \epsilon^i \bigr)
%\end{equation}
%for some constant $C_3$. Combining the above
%estimates~$(\ref{eq-bound-for-h_w})$ and~$(\ref{eq-bound-for-g_w})$
%for all $ w \in {\mathfrak X} $, and taking $ \epsilon \rightarrow 0
%$, we conclude that for every $f \in C_c^{\infty}(W)$ one has $$ \|
%f \|_{{\mathcal F}, \, max} \leqslant C_4 \| f \|_{\infty} ,$$ for
%some absolute constant $C_4$. This completes the proof of the
%theorem.
%





%\begin{remark}  {\rm Let us emphasize that any pseudo norm $\| \cdot \|$ on ${\mathcal A}$
%can be turned into a Ham$(M,\omega)$-invariant pseudo norm via the
%invariantization procedure $ \| f \| \mapsto \| f \|_{inv}$, where:
%$$ \| f \|_{inv} =  \inf \Bigl \{  \sum \|\phi_i^*  f_i \|  \ ; \ f = \sum f_i, \ {\rm and \ } \phi_i \in Ham(M,\omega)  \Bigr \} $$
%We remark that in the above definition of $\| f \|_{inv}$ the sum
%$\sum f_i$ is assumed to be finite. Note that $\| \cdot \|_{inv}
%\leq \| \cdot \|$.  Thus, if $\| \cdot \|$ is continuous in the
%$C^{\infty}$-topology, then so is $\| \cdot \|_{inv}$. Moreover if
%$\| \cdot \|'$ is a Ham$(M,\omega)$-invariant pseudo norm, then:
%$$ \| \cdot \|' \leq \| \cdot \| \Longrightarrow \| \cdot \|' \leq \| \cdot \|_{inv}$$
%In particular, the above invariantization procedure provides a
%plethora of Ham$(M,\omega)$-invariant genuine norms on ${\mathcal
%A}$, e.g., by applying it to the $\| \cdot \|_{C^k}$-norms.}
%%the above invariantization procedure on the $\| \cdot \|_{C^k}$-norms.
%%taking the homogenization of the $\| \cdot \|_{C^k}$-norms. }
%\end{remark}



%\begin{remark}  {\rm We verified that the main result in this paper (Theorem~\ref{Main-thm} above) is valid also in the case 
%of an open connected symplectic manifold of finite volume. Although we believe that the main result 
%in~\cite{OW} (Theorem~\ref{OW-theorem} above) could also be extended to this case, we did not check all the steps thoroughly. We leave 
%this for future research.}
%\end{remark}



\section{Some Open Questions and Speculations} \label{SEC:OQ}


\smallskip

\noindent{\bf Do symplectic capacities coincide on  the class of convex domains?} 
%
%O.
%Even for star-shaped subsets of R2n (n ? 2) our above capacities define
%different symplectic invariants. The earliest result known to me in this direction
%is due to Hermann [He]. He constructed for every n ? 2 star-shaped
%Reinhardt-domains in R2n with arbitrarily small volume and hence arbitrarily
%small Gromov width whose displacement energy is bounded from below by
%1. For the estimate of the displacement energy he uses a remarkable result of
%Chekanov [C] who showed that the displacement energy of a closed Lagrangian
%submanifold of R2n is positive.
As mentioned above, since the time of Gromov's original work, a variety of symplectic capacities have been constructed and the
relations between them often lead to the discovery of surprising connections between symplectic geometry and Hamiltonian dynamics.
In the two-dimensional case, Siburg~\cite{Sib} showed that any symplectic capacity of  a compact connected domain with smooth boundary $\Omega \subset {\mathbb R}^2$ equals its Lebesgue measure. 
In higher dimensions symplectic capacities do not coincide in general. 
A %symplectic hedgehog-type 
theorem  by Hermann~\cite{Her} states that for any $n \geq 2$ there
is a bounded star-shaped %Reinhardt- 
domain $S \subset {\mathbb R}^{2n}$  with
cylindrical capacity $\overline c(S) \geq 1$, and arbitrarily small Gromov
radius $\underline c(S)$.  Still, for large classes of sets in
${\mathbb R}^{2n}$, including ellipsoids, polydiscs and convex Reinhardt
domains, all symplectic capacities coincide~\cite{Her}. 
In~\cite{V} Viterbo showed that for any bounded convex subset $\Sigma$
 of ${\mathbb R}^{2n}$ one has  $\overline c(\Sigma) \leq 4n^2 \underline c(\Sigma)$. 
Moreover, one has (see~\cite{Her,H2,V}) the following: 
\begin{conjecture} \label{conj-all-cap-coincide}
For any convex domain $\Sigma$ in
${\mathbb R}^{2n}$ one has $\underline c(\Sigma) = \overline c(\Sigma)$.
\end{conjecture}
This conjecture is particularly challenging due to the scarcity of examples of convex domains in which capacities have been  computed. 
Moreover, note that Conjecture~\ref{conj-all-cap-coincide} is stronger than Viterbo's conjecture (Conjecture~\ref{iso-per-conj} above), as the latter holds trivially for
the Gromov radius.

\smallskip

A somewhat more modest question in this direction is whether Conjecture~\ref{conj-all-cap-coincide} holds asymptotically, i.e., whether there is an absolute constant $A$ such that for any convex domain $K \subset
{\mathbb R}^{2n}$ one has $\overline c(K) \leq A \, \underline c(K)$.  It would be interesting to explore whether methods from asymptotic geometric analysis can be used to answer this question. 
%Interpolating ideas from~\cite{AMO}, a possible strategy toward answering this question would be to use tools from asymptotic geometric analysis  to show 
%that the linearized Gromov radius given by $$\underline c^{lin}(U) =   \sup \{ \pi r^2 \, | \,  {\rm there \ exists \ }  S \in {\rm Sp}(2n) \ {\rm with} \ SB^{2n}(r) \subseteq U \},$$  satisfies a reverse Brunn-Minkowski inequality \`a la Theorem~\ref{RBMIneq} above.
\bigskip

\noindent{\bf Are Hanner polytopes in fact symplectic balls in disguise?}
Recall that Mahler's conjecture states that the minimum possible Mahler volume is attained by a hypercube. 
%(using the language of convex bodies)  that the minimum of the volume product of a centrally symmetric convex body and its polar is attained for the hypercube. 
It is interesting to note that the corresponding product configuration, when looked at through symplectic glasses,  is in fact a Euclidean ball in disguise. More precisely, it was proved in \S 7.9 of~\cite{Schle} (cf. Corollary 4.2 in~\cite{LMS})  that the interior of the product of a hypercube $Q \subset {\mathbb R}^n_q$ and its dual body, the cross-polytope $Q^{\circ} \subset {\mathbb R}^n_p$, is symplectomorphic to the interior of a Euclidean ball $B^{2n}(r)  \subset {\mathbb R}^n_q \times  {\mathbb R}^n_p$ with the same volume.
On the other hand, as mentioned in Section~\ref{SEC:MAHLER} above, if Mahler's conjecture holds, then there are other minimizers for the Mahler volume  aside of the hypercube. 
For example, consider the class of Hanner polytopes. A $d$-dimensional centrally symmetric polytope $P$ is
a Hanner polytope if either $P$ is one-dimensional (i.e., a symmetric interval), or $P$ 
%i) the dimension d is at most 1, or
is the free sum or direct product of two (lower dimensional) Hanner
polytopes $P_1$ and $P_2$.
%Recall that any line segment is a Hanner polytope, and that a d-polytope $P$ with $d > 1$ is a Hanner polytope if it can be written as
%the cartesian product of two Hanner polytopes or as the polar of a Hanner polytope.
%which are convex polytopes constructed recursively by applying Cartesian products and free sums to centered line segments in arbitrary order. 
Recall that the free sum of two polytopes, $P_1 \subset {\mathbb R}^n$, $P_2 \subset {\mathbb R}^m$ is  a $n+m$ polytope defined by $P_1 \oplus P_2 = {\rm Conv} ( \{ P_1 \times \{0\} \}  \cup \{  \{0\} \times P_2 \}) \subset {\mathbb R}^{n+m}$. 
It is not hard to check (see e.g.~\cite{SR}) that  the volume product of the cube is the same as that of Hanner polytopes. Thus every Hanner polytope is also a candidate for a minimizer of the volume
product among symmetric convex bodies. In light of the above mentioned result from~\cite{Schle}, a natural question is the following: %this naturally leads to the following: % question:
%
%%Cartesian product and direct sum operations. 
%%
%%More precisely, a Hanner polytope is obtained by successively applying Cartesian products and free sums to centered line segments in arbitrary order. 
%
%
%\smallskip
%
%If Mahler's conjecture is true, then these polytopes are the minimiz-
%ers of the Mahler volume among centrally symmetric convex bodies. In light of Theorem 7
%above, this leads to the following question % interesting question from the symplectic point of view:
\begin{question} Is every Hanner polytope a symplectic image of a Euclidean ball?
\end{question}
More precisely, is the interior of every Hanner polytope  symplectomorphic to the interior of a Euclidean ball with the same volume?
A negative answer to this question would give a counterexample to Conjecture~\ref{conj-all-cap-coincide} above, since it would show that the Gromov radius must be different from the Ekeland-Hofer-Zehnder capacity. 
%A positive answer, on the other hand, would  give more support  to the approach of tackling Mahler's conjecture using symplectic techniques. 

%\smallskip
%
%the structure of a Hanner polytope may be much more complicated
%than that of the cube

%
%that the volume product of the cube is the same as that of Hanner
%polytopes. Thus every Hanner polytope is also a candidate for a minimizer of the volume
%product among symmetric convex bodies. It was also shown in [26, 35] that Hanner polytopes
%are the only possible minimizers in the class of unconditional bodies.

\bigskip




%\noindent{\bf A variation on Viterbo's conjecture.}
%%\noindent{\bf A quantitative refinement to the Weinstein conjecture.}
%The above mentioned result of Hermann~\cite{Her} implies that there are starshaped domains in ${\mathbb R}^{2n}$ with arbitrarily small volume and large capacity,
%and hence Viterbo's conjecture fails for this class. A possible way to try and extend Viterbo's conjecture  beyond the convex category is the following:  
%Let $\Sigma \subset {\mathbb R}^{2n}$ be a domain with smooth boundary, and set $l_1(\partial \Sigma)$ to be the infimum of the actions of all closed characteristics on the boundary $\partial \Sigma$. If $\partial \Sigma$ has no closed characteristics, we set $l_1(\partial \Sigma)=0$.
%Recall that for the Ekeland-Hofer-Zehnder capacity on the class of convex domains in ${\mathbb R}^{2n}$, 
%%Recall that for the class of convex domains in ${\mathbb R}^{2n}$, and for the Ekeland-Hofer-Zehnder capacity, 
%Viterbo's conjecture states that $l^n_1(\partial \Sigma)  \leq  n! \, {\rm Vol}(\Sigma)$.
%This naturally leads to the following isosystolic-type question:
%%The following isosystolic-type question is a way to generalize a special case of Viterbo's conjecture, when restricted to the Ekeland-Hofer-Zehnder capacity, behind the 
%%convex category:
%\begin{question} Does $l^n_1(\partial \Sigma)  \leq n! \, {\rm Vol}(\Sigma)$ for any smooth  body  $ \Sigma \subset {\mathbb R}^{2n}$?  
%\end{question}
%Note that without the convexity assumption, the minimal action of closed characteristics on the boundary is not necessarily a symplectic capacity. 
%This question can also be translated to the setting of closed contact manifolds in terms of finding the exact relation between the minimal action 
%of a closed Reeb orbit and the volume of the underlying contact manifold (see~\cite{Hu3}).
%%It would be interesting to explore whether techniques from Riemannian geometry 
%%developed to compare the volume of a Riemannian manifold with the 
%%1-systole (see e.g., [....]), could be adjusted and applied in the above Hamiltonian setting. 
%%%exact relation between the minimal action of a closed Reeb orbit and the volume of the underlying contact manifold.
%%%The Hanner polytopes are the unit balls of a family of finite-dimensional Banach spaces called Hanner spaces, which are  %The Hanner spaces are the 
%%%spaces that can be built up from one-dimensional spaces by $\ell_1$ and $\ell_\infty$ combinations.
%
%
%\bigskip

\noindent{\bf Symplectic embeddings of Lagrangian products.}
Since Gromov's work~\cite{Gr}, questions about symplectic embeddings have lain at the heart
of symplectic geometry (see e.g.,~\cite{B1,B3,Gu,Hu2, LMS, Mcd2,McPol,McSch, Schl1,Schle}). %,Tr}).
  These questions are usually notoriously difficult, and up to date most results concern  only the embeddings of balls, ellipsoids and polydiscs. 
Note that even for this simple class of examples, only 
recently  has it  become possible to specify exactly when a four-dimensional ellipsoid is embeddable in a ball (McDuff and Schlenk~\cite{McSch}), or in another four-dimensional ellipsoid (McDuff~\cite{Mcd2}).
For some other related works we refer the reader to~\cite{BH,CCFHR,CGK,FM, HK,HL,Op1}.
\smallskip

Since symplectic capacities can naturally be used to detect symplectic embedding obstructions, and in light of the results mentioned in Section~\ref{SEC:MAHLER} (in particular, Theorem~\ref{Main-Theorem-From-AKO}), 
it is only natural to try to extend the above list of currently-known examples, and study symplectic embeddings of convex ``Lagrangian products" in 
the classical phase space. The main advantage of this class of bodies is that the action spectrum can be computed via billiard dynamics. This 
property would presumably make it easier to compute or estimate the Ekeland-Hofer capacities~\cite{EH}, 
or Hutchings' embedded contact homology capacities~\cite{Hu1,Hu2}, in this setting. 
A natural first step in this direction would be to consider the embedding of the Lagrangian product of two balls into a Euclidean ball.
More precisely, 
%interesting to 
%As obstructions for embedding usually come 
%
%
%To obtain lower bounds, I will compute the Ekeland-Hofer capacities of 1  2
%using the relation between closed characteristics of 1  2 and periodic Finsler billiard trajectories
%studied in [9], thus obtaining obstructions for embedding.
%
%
%study symplectic embedding questions for certain ``Lagrangian products" in the classical phase space,
%%
%
%I propose to break out of the narrow borders of the family of currently-known examples, and for the first time study symplectic embedding 
%questions for certain ``Lagrangian products" in the classical phase space, namely domains %in the phase space 
%of the form $\Sigma_1 \times \Sigma_2 \subset {\mathbb R}^n_q \times {\mathbb R}^n_p$. %, where $\Sigma_2$. If $\Sigma_2$ is a convex domain,
%e that if $\Sigma_2$ is a convex domain, then the closed characteristics on 
%More preciesly, when $\Sigma_2$ is a convx domain, I plan to use the relation between generalized closed characteristics of $\Sigma_1 
%\times \Sigma_2$  and periodic Finsler billiard trajectories studied in~\cite{AAO2}, togehter with 
%The first case I plan to study is the symplectic embedding of $\Sigma_1 \times \Sigma_2$ into the Euclidean open ball $B^{2n}[a]$ of capacity $a$.
%More precisely:
%\begin{question}
to study the  function $\sigma: {\mathbb N} \rightarrow {\mathbb R}$ defined by 
$$ \sigma(n) = \inf  \bigl \{a \ |  B^n_q(1)  \times B^n_p(1)  \stackrel{\rm symp} \hookrightarrow B^{2n}(a)  \bigr \}.$$
%To the best of the author's knowledge, 
%\end{question}
%Surprisingly enough, this question is open  even  when $\Sigma_1 \times \Sigma_2 = B_q^n \times B^n_p$ is a Lagrangian product of two balls.
%I plan to obtain upper bounds for $\sigma(n)$ using symplectic wrapping and  folding techniques (see e.g.,~\cite{Schle}). 
%To obtain lower bounds, I plan to compute the Ekeland-Hofer capacities of $\Sigma_1 \times \Sigma_2$
%using 
%the relation between  closed characteristics of $\Sigma_1 
%\times \Sigma_2$  and  Finsler billiard trajectories studied in~\cite{AO2},
%thus obtaining embedding obstructions.
To the best of the author's knowledge, the value of $\sigma(n)$ is unknown already for the case  $n=2$. 


\bigskip



\noindent{\bf Acknowledgement:}
 I am  deeply indebted to Leonid Polterovich for generously sharing his insights and perspective on topics related to this paper, as well as for many inspiring conversations throughout the years. I have also benefited significantly from an ongoing collaboration with Shiri Artstein-Avidan, I am grateful to her for many stimulating and enjoyable hours working together. I would also like to thank Felix Schlenk and Leonid Polterovich for their valuable comments on an earlier draft of this paper.
%
%
% I am  deeply indebted to Leonid Polterovich for generously sharing his insights and perspective on topics related to this paper, as well as for many inspiring conversations throughout the years. I have also benefited enormously from the ongoing collaboration with Shiri Artstein-Avidan, working with her is as stimulating as it is enjoyable.
%%
%%
% %I am  deeply indebted to Leonid Polterovich for generously sharing his insights and perspective on topics related to this paper, as well as for many conversations throughout the years which
% %have broadened my mathematical horizons considerably, and were an infinite source of inspiration. 
%%I was also benefited enormously from the ongoing collaboration with Shiri Artstein-Avidan, ......
%Finally, I would like to thank Felix Schlenk and Leonid Polterovich for their valuable comments on a preliminary version of this paper.

%
% an earlier version of this paper and suggesting

%\section{References}
%
%It follows a list of references...


%\frenchspacing
\begin{thebibliography}{7}

\bibitem{APB}  \'Alvarez-Paiva, J. C., Balacheff, F. {\it Optimalit\'e systolique infinit\'esimale de l'oscillateur harmonique,} S\'eminaire de Th\'eorie Spectrale et G\'eom\'etrie 27 (2009), 11--16.

\bibitem{APB1}  \'Alvarez-Paiva, J. C., Balacheff, F. {\it Contact geometry and isosystolic inequalities,} to appear in Geom. Funct. Anal. Preprint: arXiv:1109.4253.


\bibitem{ABKS} Akopyan, A.V., Balitskiy, A. M., Karasev, R. N., Sharipova, A. {\it Elementary results in non-reflexive Finsler billiards,} Preprint: arXiv:1401.0442.



\bibitem{Ar} Arnold, V. I. {\it Sur la g\'{e}om\'{e}trie diff\'{e}rentielle des groupes
de Lie de dimension infinie et ses applications \`{a}
l'hydrodynamique des fluides parfaits}, (French) Ann. Inst. Fourier
(Grenoble) 16 1966 fasc. 1 319--361.


%\bibitem{AK} Arnold, V. I., Khesin, B. A. {\it Topological Methods in
%Hydrodynamics}, Applied Mathematical Sciences, 125, Springer-Verlag,
%New York, 1998.

\bibitem{AM} Albers, P.,  Mazzucchelli, M. {\it Periodic bounce orbits of prescribed energy, }
Int. Math. Res. Not. IMRN 2011, no. 14, 3289--3314. 

\bibitem{AMO} Artstein-Avidan, S., Milman, V., Ostrover Y. {\it  The M-ellipsoid, Symplectic Capacities and Volume}, Comment. Math. Helv. 83 (2008), no. 2, 359--369.  %Commentarii Mathematici Helvetici, Vol. 83, Issue 2, 2008.


\bibitem{AAO1} Artstein-Avidan, S., Ostrover Y. {\it  Bounds for Minkowski billiard trajectories in convex bodies,} 
Intern. Math. Res. Not. (IMRN) (2012) doi:10.1093/imrn/rns216.


\bibitem{AKO} Artstein-Avidan, S., Karasev, R., Ostrover, Y. {\it From symplectic measurements to the Mahler
conjecture,} to appear in Duke Math J., Preprint. arXiv: 1303.4197.

\bibitem {B} Banyaga, A.  {\it Sur la structure du groupe des
diff\'eomorphisms qui pr\'eservent une forme symplectique}, Comment.
Math. Helv. {\bf 53} (1978), no.2, 174--227.

\bibitem{BG} Benci, V., Giannoni, F. {\it Periodic bounce trajectories with a low number of bounce points,}
Ann. Inst. H. Poincar\'e Anal. Non Lin\'eaire 6 (1989), no. 1, 73--93. 

\bibitem{BB} Bezdek, D.,  Bezdek, K. {\it Shortest billiard trajectories,} 
Geom. Dedicata 141 (2009), 197--206. 

\bibitem{B1} Biran, P. {\it Symplectic packing in dimension 4,} Geom. Funct. Anal. 7 (1997), no. 3, 420--437. 

%\bibitem{B2} Biran, P. {\it A stability property of symplectic packing,} Invent. Math. 136 (1999), no. 1, 123--155. 

\bibitem{B3} Biran, P. {\it  Lagrangian barriers and symplectic embeddings,} Geom. Funct. Anal. 11 (2001), no. 3, 407--464.



\bibitem{Bl} Blaschke, W. {\it \"Uber affine Geometrie VII: Neue Extremeigenschaften
von Ellipse und Ellipsoid}, Ber. Verh. S\"achs. Akad. Wiss. Leipzig,
Math.-Phys. Kl {\bf 69} (1917) 306--318, Ges. Werke {\bf 3} 246--258.

\bibitem{BH} Buse, O., Hind, R. {\it Ellipsoid embeddings and symplectic  packing  stability},  Compos. Math. 149 (2013), no. 5, 889--902. 



\bibitem{BM} Bourgain, J., Milman, V. D. {\it New volume ratio properties for convex
symmetric bodies in ${\mathbb R}^n$,} Invent. Math. 88 (1987), no.
2, 319--340.

\bibitem{BO} Buhovsky L., Ostrover, Y., {\it Bi-invariant Finsler metrics on the group of Hamiltonian diffeomorphisms}, 
Geom. Funct. Anal. 21 (2011), no. 6, 1296--1330. 

\bibitem{Chek} Chekanov, Yu. V. {\it Lagrangian intersections, symplectic energy, and areas of holomorphic curves,} Duke Math. J. 95 (1998), no. 1, 213--226.



\bibitem{CCFHR} Choi, K., Cristofaro-Gardiner, D., Frenkel, D., Hutchings, M., Ramos, V.G.B {\it Symplectic embeddings into four-dimensional concave toric domains,} arXiv:1310.6647.



\bibitem{CHLS} Cieliebak, T., Hofer, H., Latschev, J., Schlenk
F. {\it Quantitative symplectic geometry,}  Dynamics, ergodic
theory, and geometry, 1-44,  Math. Sci. Res. Inst. Publ., 54,
Cambridge Univ. Press, Cambridge 2007.

\bibitem{Cl} Clarke, F. H. {\it Periodic solutions to Hamiltonian inclusions, } J. Differential Equations 40 (1981), no. 1, 1--6. 

\bibitem{CGK} Cristofaro-Gardiner, D., Kleinman, A. {\it Ehrhart polynomials and symplectic embeddings of ellipsoids}, arXiv:1307.5493.

\bibitem{Eke}  Ekeland, I. {\it Convexity Methods in Hamiltonian Systems,}  Ergeb. Math. Grenzgeb. 19, Springer, Berlin, 1990.

\bibitem{EH} Ekeland, I. and Hofer, H. {\it Symplectic topology
and Hamiltonian dynamics,} Mathematische Zeitschrift, {\bf 200}
(1989), no. 3, 355-378.



\bibitem{Eli} Eliashberg, Y. {\it Symplectic topology in the nineties,}  Symplectic geometry. Differential Geom. Appl. 9 (1998), no. 1-2, 59--88.




\bibitem{EliP} Eliashberg, Y., Polterovich, L. {\it Bi-invariant
metrics on the group of Hamiltonian diffeomorphisms.} Internat. J.
Math. {\bf 4} (1993), 727-738.

\bibitem{FH} Floer, A.,  Hofer, H. {\it Symplectic homology. I. Open sets in ${\mathbb C}^n$,} Math. Z. 215 (1994), no. 1, 37--88. 

\bibitem{FHW} Floer, A.,  Hofer, H., Wysocki, K.  {\it  Applications of symplectic homology. I,}  Math. Z. 217 (1994), no. 4, 577--606


\bibitem{FGS} Frauenfelder, U., Ginzburg, V., Schlenk, F. {\it Energy capacity
inequalities via an action selector},  Geometry, spectral theory,
groups, and dynamics, 129-152, Contemp. Math., 387, Amer. Math.
Soc., Providence, RI, 2005.

\bibitem{FM}  Frenkel, D.,  M\"uller, D. {\it  Symplectic embeddings of 4-dimensional ellipsoids into cubes}, arXiv:1210.2266.

\bibitem{Gh} Ghomi, M. {\it Shortest periodic billiard trajectories in convex bodies, }
Geom. Funct. Anal. 14 (2004), no. 2, 295--302. 

\bibitem{GPV} Giannopoulos, A., Paouris, G., Vritsiou, B. {\it The isotropic position and the reverse Santal\'o inequality,}
to appear in Israel J. Math. Preprint, arXiv:1112.3073.

\bibitem{GMR} Gordon, Y., Meyer,  M.,  Reisner, S. {\it Zonoids with minimal
volume product -- a new proof,} Proc. Amer. Math. Soc.
104 (1988), no. 1, 273--276.

\bibitem{Gr} Gromov, M. {\it Pseudoholomorphic curves in symplectic manifolds}, Invent. Math. 82 (1985), no. 2, 307-347.

\bibitem{Gu} Guth, L. {\it  Symplectic embeddings of polydisks,} Inven. Math. 172 (2008), 477--489.

\bibitem{GT} Gutkin, E., Tabachnikov, S. {\it Billiards in Finsler and Minkowski geometries,} J. Geom. Phys. 40 (2002), no. 3-4, 277--301.

\bibitem{Her} Hermann, D. {\it Non-equivalence of symplectic capacities for open
sets with restricted contact type boundary}. Pr\'epublication
d'Orsay num\'ero 32 (29/4/1998).

\bibitem{HK} Hind, R., Kerman, E. {\it New obstructions to symplectic embeddings,} preprint:  arXiv:0906.4296.

\bibitem{HL}  Hind, R., Lisi, S. {\it Symplectic embeddings of polydisks,} To appear in Selecta Mathematica.  Preprint: arXiv:1304.3065

\bibitem{H} Hofer, H. {\it On the topological properties of symplectic
maps,} Proc. Roy. Soc. Edinburgh Sect. A 115, 25-38 (1990).

\bibitem{H2} Hofer, H. {\it Symplectic capacities}, 
Geometry of low-dimensional manifolds, 2 (Durham, 1989), 15-34, 
London Math. Soc. Lect. Note Ser., 151, Cambridge Univ. Press,  1990. 

\bibitem{HZ1} Hofer, H., Zehnder, E. {\it A new capacity for symplectic manifolds,}  Analysis, et cetera, 405--427, Academic Press, Boston, MA, 1990. 


\bibitem{HZ} Hofer, H.,  Zehnder, E. {\it Symplectic invariants
and Hamiltonian dynamics}, Birkhauser Advanced Texts, Birkhauser
Verlag, 1994.


\bibitem{Hu1} Hutchings, M. {\it Quantitative embedded contact homology}, J. Differential Geom. 88 (2011), no. 2, 231--266. 


\bibitem{Hu2} Hutchings, M. {\it Recent progress on symplectic embedding problems in four dimensions,} Proc. Natl. Acad. Sci. USA 108 (2011), no. 20, 8093--8099.

\bibitem{Hu3} Hutchings, M. {Some open problems on symplectic embeddings and the Weinstein conjecture}, 
http://floerhomology.wordpress.com/2011/09/14/open-problems/. 

\bibitem{IrirKei1} Irie, K. {\it Symplectic capacity and short periodic billiard trajectory,} Math. Z. 272 (2012), no. 3--4, 1291--1320.

\bibitem{IrieKei2} Irie, K. {\it Periodic billiard trajectories and Morse theory on loop spaces,}  Preprint: arXiv:1403.1953.

\bibitem {Kim} Kim, J. {\it Minimal volume product near Hanner polytopes,} J. Funct. Anal. 266 (2014), no. 4, 2360--2402. 


\bibitem{Kun} K\"unzle, A. F. {\it Singular Hamiltonian systems and symplectic capacities}, Singularities and differential equations (Warsaw, 1993), 171--187, 
Banach Center Publ., 33, Polish Acad. Sci., Warsaw, 1996. 

\bibitem {Ku} Kuperberg, G. {\it From the Mahler conjecture to Gauss linking
integrals,} Geom. Funct. Anal., 18, no. 3, (2008),
870--892.

\bibitem{L} Lalonde, F. {\it Energy and capacities in symplectic
topology,} in: Geometric topology (Athens, GA, 1993), 328-374, AMS/IP
Stud. Adv. Math., {\bf 2.1}, Amer. Math. Soc., Providence, RI, 1997.

\bibitem{LM}  Lalonde, F., McDuff, D. {\it The geometry of symplectic energy,}  Ann. of Math. (2) 141 (1995), no. 2, 349--371

\bibitem{LM1}Lalonde, F., McDuff, D. {\it Hofer's $L^{\infty}$-geometry: Energy and
stability of Hamiltonian flows}, parts I, II, Invent. Math. 122
(1995), 1--33, 35--69.

\bibitem{LMT} Landry, M.,  McMillan, M., Tsukerman, E. {\it  On symplectic capacities of toric domains,} Preprint: arXiv:1309.5072.

\bibitem{LMS} Latschev, J., McDuff, D., Schlenk, F. {\it The Gromov width of 4-dimensional tori}, arXiv:1111.6566.


\bibitem{Lu} Lu, G. {\it  Gromov-Witten invariants and pseudo symplectic capacities,}  Israel. J.
Math. 156 (2006), 1--63.


\bibitem{Ma} Mahler, K. {\it Ein \"Ubertragungsprinzip f\"ur konvexe
Korper,} Casopis Pyest. Mat. Fys. 68, (1939), 93--102.

%\bibitem{Mcd} McDuff, D. {\it Symplectic topology and capacities.} in
%Prospects in mathematics (Princeton, NJ, 1996), 69-81, Amer. Math.
%Soc., Providence, RI, 1999.

\bibitem{Mcd1} McDuff, D. {\it Geometric variants of the Hofer norm,}  J. Symplectic Geom. 1  (2002),  no. 2, 197--252.



\bibitem{Mcd2} McDuff, D. {\it The Hofer conjecture on embedding symplectic ellipsoids,}  J. Diff. Geom. 88 (2011), no. 3, 519--532. 


\bibitem{McPol} McDuff, D.,  Polterovich, L. {\it Symplectic packings and algebraic geometry. With an appendix by Yael Karshon,} Invent. Math. 115 (1994), no. 3, 405--434.

\bibitem{McSal} McDuff, D., Salamon, D. {\it Introduction to Symplectic Topology,} Second edition. Oxford Mathematical Monographs. The Clarendon Press, Oxford University Press, New York, 1998.

\bibitem{McSch} McDuff, D.,  Schlenk, F. {\it The embedding capacity of 4-dimensional symplectic ellipsoids,}
Ann. of Math. (2) 175 (2012), no. 3, 1191--1282. 


\bibitem{ME1} Meyer, M. {\it Une caract\'erisation volumique de certains espaces
norm\'es de dimension finie,} Israel J. Math. 55 (1986), no. 3,
317--326.

\bibitem{Mil} Milman, V.D. {\it An inverse form of the Brunn-Minkowski inequality with applications to the local theory of normed spaces.}
C. R. Acad. Sci. Paris S\'er. I Math. 302 (1986), no. 1, 25--28.

\bibitem{Milm1317}  Milman, V.D. {\it Isomorphic symmetrizations and geometric
inequalities.} in: Geometric aspects of functional analysis (1986/87),
107--131, Lecture Notes in Math., 1317, Springer, Berlin, 1988.

\bibitem{MilSch}  Milman, V.D., Schechtman, G. {\it Asymptotic Theory of Finite Dimensional Normed Spaces,} 
Lectures Notes in Math. 1200, Springer, Berlin (1986).

\bibitem{Naz} Nazarov, F. {\it The H\"{o}rmander proof of the Bourgain-Milman theorem}, in: Geometric aspects of functional analysis, 335--343, 
Lecture Notes in Math., 2050, Springer, Heidelberg, 2012. 

\bibitem{NPRZ} Nazarov, F., Petrov, F., Ryabogin, D., Zvavitch,
A. {\it A remark on the Mahler conjecture: local minimality of the unit cube,} Duke Math. J. 154 (2010), no. 3, 419--430. 

\bibitem{Oh} Oh, Y-G. {\it Chain level Floer theory and Hofer's geometry of the Hamiltonian diffeomorphism
group,} Asian J. Math. 6 (2002), no. 4, 579--624.

%\bibitem{Op} Opshtein, E. {\it Singular polarizations and ellipsoid packings,}  Int. Math. Res. Not.  2013, no. 11, 2568--2600. 

\bibitem{Op1} Opshtein, E. {\it Symplectic packings in dimension 4 and singular curves, } arXiv:1110.2385.

\bibitem{OW} Ostrover, Y., Wagner, R., {\it On the extremality of Hofer's metric on the group
of Hamiltonian diffeomorphisms}, Int. Math. Res. Not. 35 (2005), 2123--2141. 

\bibitem{Pi1} Pisier, G. {\it The Volume of Convex Bodies and Banach Space
Geometry.} Cambridge University Press, Cambridge, (1989).

\bibitem{P} Polterovich, L. {\it Symplectic displacement energy for Lagrangian submanifolds,} 
Ergodic Theory Dynam. Systems 13 (1993), no. 2, 357--367.

\bibitem{P1} Polterovich, L. {\it The Geometry of the Group of Symplectic Diffeomorphisms},
Lectures in Mathematics ETH Z\"{u}rich. Birkh\"{a}user Verlag,
Basel, 2001.



\bibitem{R1} Reisner, S. {\it Zonoids with minimal volume product,} Math.
Z. 192 (1986), no. 3, 339--346.

\bibitem{R2} Reisner, S. {\it Minimal volume-product in Banach spaces with a
1-unconditional basis,} J. London Math. Soc. 36 (1987),  no.1, 126--136.

\bibitem{RSW} Reisner, S., Sch\"utt, C., Werner, E. {\it Mahler's conjecture and curvature,}
Int. Math. Res. Not.  2012, no. 1, 1--16. 

%\bibitem{Rock} Rockafellar, R. T. {\it The theory of subgradients and its applications to problems of optimization,} Convex and nonconvex functions. R \& E, 1. Heldermann Verlag, Berlin, 1981.

\bibitem{RogShe} Rogers, C.A., Shephard, C. {\it The difference
body of a convex body.} Arch. Math. 8 (1957), 220--233.


\bibitem{SR} Saint Raymond, J. {\it Sur le volume des corps convexes
sym\'etriques,} in:  Initiation Seminar on Analysis: G. Choquet--M.
Rogalski--J. Saint-Raymond, 20th Year: 1980/1981, Exp. No. 11, Publ. Math. Univ. Pierre et Marie Curie, 46, Univ. Paris VI,
Paris, 1981.


\bibitem{Sa} Santal\'o, L.A. {\it Un invariante afin para los cuerpos convexos de espacio de $n$ dimensiones,}
Portugal. Math {\bf 8} (1949) 155--161.

\bibitem{Schle} Schlenk. F. {\it Embedding Problems in Symplectic Geometry,}  de Gruyter Expositions in Mathematics, 40, Berlin, 2005.

\bibitem{Schl1} Schlenk, F. {\it Symplectic embeddings of ellipsoids,}  Israel J. Math. 138 (2003), 215--252.


\bibitem{Sch} Schwarz, M. {\it On the action spectrum for closed symplectically
aspherical manifolds}, Pacific J. Math. 193 (2000), 1046--1095.


\bibitem{Sib} Siburg, K. F. {\it Symplectic capacities in two dimensions}, Manuscripta Math. 78 (1993), no. 2, 149--163. 

\bibitem{St} Stancu, A. {\it Two volume product inequalities and their applications, } Canad. Math. Bull. 52 (2009), no. 3, 464--472. 

\bibitem{Tao} Tao. T. % {\it Open question: the Mahler conjecture on convex bodies, } %terrytao.wordpress.com/2007/03/08/open-problem-the-mahler-conjecture-on-convex-bodies/, 2007.
{\it Structure and Randomness. Pages from Year One of a Mathematical Blog,} American Mathematical Society, Providence, RI, 2008.

%\bibitem{Tr} Traynor, L. {\it Symplectic packing constructions,}  J. Diff. Geo. 42 (1995), no. 2, 411--429. 

\bibitem{V} Viterbo, C. {\it Metric and isoperimetric problems in symplectic geometry.} J. Amer. Math. Soc. 13 (2000), no. 2, 411--431.

\bibitem{V1} Viterbo, C. {\it Symplectic topology as the geometry of
generating functions,} Math. Ann. 292, 685--710 (1992).



%
%
%\bibitem{BaRh} Babu{\v{s}}ka, I.,  Rheinboldt,  W. C.,
%Error Estimates for Adaptive Finite Element Computations,
%\emph{SIAM J. Numer. Anal.}  \textbf{15} (1978), 736--754.
%
%\bibitem{FrQu}
%Freedman, M. H.,  Quinn, F., \emph{Topology of 4-manifolds}.
%Princeton Mathematical Series~39, Princeton University
%Press, Princeton, NJ, 1990.

\end{thebibliography}


\end{document}
