\chapter{Literature Overview}
\label{chap:literature-overview}

This chapter surveys prior work surrounding the Viterbo Volume--Capacity Conjecture\cite{Viterbo2000} and closely
related directions. The intended reader knows standard material in symplectic/contact geometry;
definitions are given as concise reminders, with emphasis on normalization and notation choices.

\section*{Audience and Scope}

- Audience: MSc-level mathematicians with background in differential/symplectic/contact geometry,
  and experts seeking a quick map of results and techniques.
- Goal: collect the main lines of work around the Viterbo conjecture, indicate known cases and
  bounds, and flag adjacent topics (e.g., Mahler) that inform the landscape.

\section*{Notation and Conventions (refresher)}

- Ambient space: $(\mathbb{R}^{2n},\omega_0)$ with the standard symplectic form
  $\omega_0 = \sum_{j=1}^n dx_j \wedge dy_j$.
- Liouville one-form: $\lambda_0 = \tfrac{1}{2}\sum_{j=1}^n\bigl(x_j\,dy_j - y_j\,dx_j\bigr)$; some
  authors use $\sum_j x_j\,dy_j$. The two choices differ by an exact form, hence yield the same
  action on closed characteristics (standard), so capacity normalizations are unaffected\cite{CHLS2007}.
- Convex bodies: $K \subset \mathbb{R}^{2n}$ compact, convex, with nonempty interior;\quad
  volume $\mathrm{vol}(K)$ is $2n$-dimensional Lebesgue measure.
- Ellipsoids and balls: $E(a_1,\dots,a_n)=\{\sum_j (\pi |z_j|^2/a_j)\le 1\}$;\quad
  $B^{2n}(R)=E(\pi R^2,\dots,\pi R^2)$ so that $\mathrm{vol}\bigl(B^{2n}(R)\bigr)=\frac{(\pi R^2)^n}{n!}$.
- Capacities: a symplectic capacity $c$ satisfies monotonicity, conformality ($c(\lambda X)=\lambda^2 c(X)$),
  and normalization $c\bigl(B^{2n}(1)\bigr)=\pi$.

\section{Symplectic Capacities (reminders)}

- Gromov width $c_B$: supremal $\pi r^2$ of a symplectically embedded ball $B^{2n}(r)$.
- Cylindrical capacity $c_Z$: infimal $\pi R^2$ such that $X$ embeds into $Z^{2n}(R)=B^2(R)\times\mathbb{R}^{2n-2}$.
- Hofer--Zehnder capacity $c_{\mathrm{HZ}}$: variational supremum over admissible Hamiltonians.
- Ekeland--Hofer--Zehnder $c_{\mathrm{EHZ}}$: first Ekeland--Hofer min--max value; equals $c_{\mathrm{HZ}}$ on convex bodies\cite{CHLS2007}.
- Sequences: higher Ekeland--Hofer capacities $c_k^{\mathrm{EH}}$, Viterbo/symplectic-homology capacities
  $c_k^{\mathrm{SH}}$ for star-shaped domains (with $c_1^{\mathrm{SH}}=c_{\mathrm{EHZ}}$ in convex settings; see e.g.\ \cite{Irie2022}).
- In dimension $4$, embedded contact homology (ECH) capacities provide a sharp sequence of embedding
  obstructions and satisfy volume asymptotics\cite{CristofaroGardiner2015,Hutchings2014}; they coincide with
  other capacities on special classes (e.g., many toric/Reinhardt examples), but not in full generality.

\section{The Viterbo Conjecture}

Fix a normalized capacity $c$ (e.g., $c_{\mathrm{EHZ}}$, $c_B$, $c_Z$) on subsets of $\mathbb{R}^{2n}$.
\emph{Viterbo's Volume--Capacity Conjecture} asserts the sharp inequality
\begin{equation}\label{eq:viterbo}
  \mathrm{vol}(K) \;\le\; \frac{c(K)^n}{n!}\quad\text{for all convex }K\subset\mathbb{R}^{2n},
\end{equation}
with equality for ellipsoids (hence for balls under our normalization). Equivalently, among convex
bodies with fixed capacity, ellipsoids maximize volume; among convex bodies with fixed volume,
ellipsoids minimize capacity. The conjecture is independent of the particular normalized capacity,
as all such capacities are expected to be equivalent on convex bodies up to universal constants.

\paragraph{Normalization sanity check.}
For $K=B^{2n}(R)$ one has $c(K)=\pi R^2$ and $\mathrm{vol}(K)=(\pi R^2)^n/n!$, giving equality in
\eqref{eq:viterbo}.

\section{Post-2024 Update}

Recent work of Haim-Kislev and Ostrover\cite{HaimKislevOstrover2024} exhibits a counterexample to the global form of the volume--capacity conjecture in dimension four: a Lagrangian product of planar convex polygons (notably, a product of congruent pentagons) violates the expected ball-extremality for the EHZ capacity. In their normalization one finds, with $A$ the area of each pentagon,
\[ \frac{c_{\mathrm{EHZ}}(K\times T)^2}{2\,A^2} = \frac{\sqrt{5}+3}{5} > 1, \]
contradicting the ball extremizer prediction. The construction leverages explicit formulas for polytopes and Minkowski billiards; see also\ \cite{HaimKislev2019}. Variants restricted to additional symmetry (e.g., centrally symmetric bodies) and near-ball regimes remain plausible; for the latter, see Abbondandolo--Benedetti--Edtmair\cite{AbbondandoloBenedettiEdtmair2023}.

\section{Known Cases, Bounds, and Themes}

This section summarizes what is known, grouping by the nature of the result. We intentionally state
qualitative forms when exact constants vary by capacity/assumption across the literature.

\subsection{Sharp and Model Cases}

- \textbf{Ellipsoids and balls.} Equality holds in \eqref{eq:viterbo}. Capacities are easy to compute
  and coincide with the principal axes data; this calibrates the conjecture.
- \textbf{Products and polydisks.} For polydisks and certain product domains, explicit formulae for
  several capacities are available and are consistent with the conjectured sharp constant.
- \textbf{Toric/Reinhardt settings (4D).} In $\mathbb{R}^4$, convex toric/Reinhardt domains admit
  combinatorial and holomorphic-curve tools for ECH/EH capacities\cite{Hutchings2014,Wormleighton2021};
  volume asymptotics and embedding obstructions are sharp in many families, offering structured evidence.

\subsection{Dimension-Dependent Inequalities}

- \textbf{Exponential-/polynomial-type bounds.} For general convex bodies, there are dimension-dependent
  upper bounds of the form $\mathrm{vol}(K)\le C(n)\,c(K)^n$ with $C(n)$ suboptimal relative to $1/n!$.
  Successive works improved $C(n)$ across capacities and structural assumptions (e.g., central symmetry,
  zonoids), but a sharp or dimension-free characterization remains open beyond model classes; see, e.g.,
  dimension-independent comparisons via the M-ellipsoid/volume-radius framework\cite{ArtsteinAvidanMilmanOstrover2008}.
- \textbf{Symmetric classes.} Stronger bounds are known for centrally symmetric bodies in some regimes.
  Several techniques leverage polarity and dual norms (e.g., via Minkowski billiards) to tie
  $c_{\mathrm{EHZ}}$ to combinatorial or metric data of $K$ and $K^{\circ}$.

\subsection{Special Structures and Computability}

- \textbf{Minkowski billiards characterization.} For smooth convex $K$, $c_{\mathrm{EHZ}}(K)$ equals the
  minimal action of a closed characteristic, which corresponds to the length of the shortest periodic
  billiard trajectory in the norm with unit ball $K^{\circ}$. For polytopes, this reduces to bounce
  sequences on facets, enabling algorithmic approximations.
- \textbf{Capacities from symplectic homology.} For star-shaped domains, Viterbo- and SH-based capacities
  offer a filtration-driven perspective compatible with embeddings and ellipsoidal/toric examples\cite{Viterbo2000,Irie2022}.
- \textbf{ECH capacities in 4D.} ECH provides a complete sequence of obstructions in dimension four with
  precise volume asymptotics\cite{CristofaroGardiner2015}, reinforcing the volume--capacity linkage, though the first capacity alone
  is generally not sufficient to settle \eqref{eq:viterbo}.

\section{Techniques and Connections}

- \textbf{Variational methods (Ekeland--Hofer, Hofer--Zehnder).} The action functional on the loop space of
  $\partial K$ yields min--max invariants; for convex $K$, the first value coincides with
  $c_{\mathrm{EHZ}}(K)$\cite{CHLS2007}.
- \textbf{Spectral invariants and symplectic homology (Viterbo).} Positive $S^1$-equivariant symplectic homology
  encodes sequences of capacities compatible with embeddings and product operations\cite{Viterbo2000,Irie2022}.
- \textbf{Billiards and duality.} The Minkowski billiards viewpoint realizes closed characteristics as
  periodic trajectories with respect to a dual norm, tying $c_{\mathrm{EHZ}}$ to support functions and
  polarity; see e.g.\ \cite{HaimKislev2019}.
- \textbf{Convex geometry bridges (Mahler).} Symplectic reduction and duality arguments connect volume--capacity
  inequalities to the Mahler volume product problem in convex geometry\cite{ArtsteinAvidanKarasevOstrover2014}. While neither conjecture is known in
  full generality, progress often flows both ways for symmetric classes and products.
- \textbf{Embedding obstructions (ECH, EH hierarchies).} Capacity sequences constrain symplectic embeddings,
  produce sharp obstructions in $4$D, and offer asymptotic volume control\cite{McDuffSchlenk2012}.

\section{Historical Arc (high level)}

- \textbf{Origins.} The capacity framework originates in Gromov's non-squeezing and subsequent axiomatizations
  (Ekeland--Hofer; Hofer--Zehnder). Viterbo formulated the volume--capacity inequality in the 1990s/2000s as an
  isoperimetric-type principle in symplectic geometry.
- \textbf{Floer-theoretic capacities.} Viterbo's symplectic-homology capacities and later refinements connected
  the inequality to spectral invariants and filtered chain complexes, with compatibility under embeddings and
  product operations.
- \textbf{Convex bridges.} Work at the interface with convex geometry (e.g., Artstein-Avidan--Karasev--Ostrover)
  highlighted polarity/Mahler-type mechanisms and produced sharp comparisons in special classes.
- \textbf{Dimension four.} ECH capacities provided a complete obstruction sequence with volume asymptotics,
  offering a testing ground for refined statements and computations in low dimensions.

\section{Capacity Comparisons on Convex Bodies}

For convex bodies, several capacities are comparable up to absolute constants or with explicit
dimension dependence. Typical comparisons (to be instantiated with precise constants in the
references section of the thesis):

- $c_B \le c_Z$ and $c_{\mathrm{EHZ}} \le c_Z$ generally (capacity axioms); for many convex classes,
  $c_{\mathrm{EHZ}}$ and $c_B$ are equivalent up to universal constants.
- On centrally symmetric bodies, duality yields two-sided estimates for $c_{\mathrm{EHZ}}$ in terms of
  support functions of $K$ and its polar $K^{\circ}$ (via Minkowski billiards).
- For toric/Reinhardt domains in $\mathbb{R}^4$, $c_1^{\mathrm{SH}}$ and first ECH capacity often agree and
  track extremal projections, matching $c_Z$ in model examples.
- Near the Euclidean ball, all normalized capacities coincide on smooth convex domains\cite{AbbondandoloBenedettiEdtmair2023}.
- On convex bodies, several capacities are equivalent up to constants, with dimension-dependent and
  in some regimes dimension-independent bounds\cite{ArtsteinAvidanMilmanOstrover2008}.

\section{Relations to Mahler's Volume Product}

Let $K \subset \mathbb{R}^{2n}$ be centrally symmetric with polar $K^{\circ}$. The Mahler product
is $\mathcal{M}(K)=\mathrm{vol}(K)\,\mathrm{vol}(K^{\circ})$, conjecturally minimized by the cube (or
Hanner polytopes) among symmetric bodies. Volume--capacity bounds interact with $\mathcal{M}(K)$ in
several ways:

- Lower bounds on $\mathcal{M}(K)$ translate to upper bounds on $c(K)$ for fixed $\mathrm{vol}(K)$ through
  polarity and billiard characterizations of $c_{\mathrm{EHZ}}$.
- Conversely, sharp capacity estimates for $K$ and $K^{\circ}$ constrain $\mathcal{M}(K)$.
- In special families (e.g., cubes, cross-polytopes, zonoids), the conjectured sharp constant in
  \eqref{eq:viterbo} aligns with extremal Mahler behavior.

\section{Computational and Algorithmic Aspects}

- \textbf{Polytopes via Minkowski billiards.} For a polytope $P=\{Ax\le b\}$ with facet normals in
  the rows of $A$ and support numbers $b$, periodic billiards bounce among facets. The action of a
  closed characteristic becomes a linear expression in $b$ constrained by word/combinatorial data.
  This enables search over bounce words and convex subproblems for candidate actions.
- \textbf{Toric combinatorics (4D).} For convex toric domains, lattice-path algorithms compute
  ECH/EH capacities from the moment polytope, enabling bulk checks of volume asymptotics.
- \textbf{Numerical SH proxies.} Discrete Reeb-graph approximations and persistence computations can
  provide exploratory upper/lower bounds for $c_1^{\mathrm{SH}}$ on star-shaped domains, to be
  cross-validated against billiard/ECH data.

\section{Open Directions and Heuristics}

- Tighten constants in capacity comparisons on symmetric bodies and zonoids.
- Clarify equivalence (up to universal constants) among $c_B$, $c_Z$, $c_{\mathrm{EHZ}}$ on general convex
  bodies; identify sharp families where gaps persist.
- Extend Minkowski-billiard computational methods from polytopes to smooth convex bodies with
  certified a posteriori bounds.
- In $4$D, map when ECH, EH, and SH first capacities coincide and when they differ, relative to
  projections controlling $c_Z$.

\section{Surveys and Expository Guides}

- Symplectic capacities and convex geometry surveys (overview of capacity axioms, examples, and convex links).
- Expository notes on ECH capacities (dimension four) and their volume asymptotics.
- Monographs and lecture notes on symplectic embedding problems that contextualize capacity methods.

\section{Further Literature (not yet fully digested)}

The following topics/papers are closely related by scope or technique. Items marked “from abstract”
reflect sources we have not yet verified in detail; they motivate future passes through the chapter.

- Improved constants in volume--capacity inequalities for symmetric bodies (from abstract).
- Viterbo capacities via generating functions and comparisons with EH capacities (from abstract).
- Toric domain capacities beyond Reinhardt convexity; lattice-path combinatorics (from abstract).
- Mahler connections via symplectic reduction; stability under perturbations (from abstract).
- Minkowski billiards for polytopes; combinatorial enumeration of bounce words (from abstract).

\section{Selected Reading Guide (names without full citations yet)}

The following sources offer broad coverage or keystone results; precise BibTeX will be added in a
subsequent pass.

- Viterbo: original statements of the volume--capacity inequality and symplectic-homology capacities.
- Ostrover: surveys on symplectic capacities and convex geometry connections (volume/capacity/embedding).
- Artstein-Avidan, Karasev, Ostrover: links to Mahler-type inequalities and sharp cases in convex classes.
- Cieliebak, Frauenfelder, Paternain: identification of $c_1^{\mathrm{SH}}$ with $c_{\mathrm{EHZ}}$ in convex settings.
- Hutchings: ECH capacities in dimension four and volume asymptotics; expository notes and algorithms.
- McDuff, Schlenk: symplectic embedding problems in four dimensions (context and techniques informing capacities).
- Akopyan, Karasev, and collaborators: Minkowski billiards interpretations of $c_{\mathrm{EHZ}}$ for convex bodies.

\section{Positioning of This Thesis}

- We focus on convex polytopes and star-shaped convex bodies, with an algorithmic lens on
  $c_{\mathrm{EHZ}}$ and related capacities. Computational experiments and structural observations
  in low dimensions (notably $4$D) are used to probe instances of \eqref{eq:viterbo} and to
  benchmark bounds across families.
- The chapter aims to orient the reader quickly, establish normalization choices, and supply
  a map of techniques and references that subsequent chapters leverage.
